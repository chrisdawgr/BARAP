\chapter{Eksperimental Opsætning}
Denne sektion har til formål..
\section{Valg af metoder}
Metoderne er først og fremmest udvalgt mhp. at opnå en korrespondanceanalyse af markbilleder. De udvalgte metoder er derfor anerkendte og veldokumenterede. Derudover er metoderne udvalgt mhp. at be - eller afkræfte de opstillede hypoteser omkring, hvilke nødvendige forudsætninger metoderne skal besidde for at opnå en korrespondanceanalyse af markbilleder.
\subsection{Udvalgte detektorer}
Følgende detektionsmetoder er implementeret:
\begin{itemize}
\item{Moravec\cite{moravec}. En hjørnedetektor introduceret af H. Moravec i 1980.}
\item{Harris\cite{harris}. En hjørnedetektor introduceret af C.Harris \& M.Stephens i 1988.}
\item{Difference of Gaussian\cite{SIFT}. En skala invariant blobdetektor, introduceret af D.Lowe i 2004.}
\item{Determinant of Hessian \cite{SURF}. En skala invariant blobdetektor, introduceret af H.Bay, T.Tuytelaars \& L. Van Gool. i 2006}
\end{itemize}
Detektionsmetoderne er udvalgt for at undersøge nødvendigheden af skalainvarians og, hvilken struktur, der detetekteres bedst i markbilleder.
\subsection{Udvalgte deskriptorer}
\begin{itemize}
\item{SIFT \cite{SIFT}. En deskriptor introduceret af D. Lowe i 2004}
\item{SURF \cite{SURF}. En deskriptor introduceret af  H.Bay, T.Tuytelaars \& L. Van Gool. i 2006}
\item{U-SURF\footnote{Upright-SURF, en ikke rotationsinvariant version af SURF}} 
\end{itemize}
Deskriptorene er udvalgt for undersøge nødvendigheden af rotationsinvarians.
\section{Eksperimental opsætning}
<hvilke billeder brugt>
<Kombinationer>