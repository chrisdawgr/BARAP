\chapter{Introduction} \label{sec:intro}
Future Cropping er et forsknings projekt, oprettet af Miljøstyrelsen, der i samarbejde med Datalogisk institut og institut for Plante- og Miljøvidenskab skal hjælpe landmænd med at automatisere detektion af ukrudt i marker. Projektet har til formål at nedsætte landbrugets forbrug af pesticider, ved hjælp af en drone \footnote{UAV (eng. unmanned aerial vehicle} udstyret med kamera og GPS. Dronen tager et antal overlappende luft billeder over landmandens mark, disse billeder bliver analyseret af en algoritme, der identificere, hvor ukrudtet befinder sig og derved, hvor der er behov for pesticider. Derefter oprettes et ukrudtskort, der bestemmer, hvordan de enkelte billeder korrespondere med hinanden og derved giver et fuldt overblik over, hvor ukrudtet, detekteret i de enkelte billeder, befinder sig\cite{drone}.
\section{Opgavens problemfelt} \label{subsec:felt}
Opgavens afsæt i dette projekt er bestemmelsen af, hvordan de individuelle billeder passer sammen, hvilket er det første skridt i etableringen af ukrudtskortet. Dette udføres ved at etablere korrespondancer imellem billederne, taget af dronen, hvilket er muligt da billederne overlapper hinanden. Denne teknik vil refereres til som "korrespondanceanalyse". Korrespondanceanalyse er en billedebehandlings teknik, indenfor Computer Vision og består af følgende trin. (\textit{1}) Distinktive punkter detekteres i billederne som resultat af at anvende af en række matematiske modeller på billederne. Formålet ved dette stadie er udvælge de samme punkter/objekter i de overlappende billeder. (\textit{2}) Området omkring disse fundne punkter beskrives af en deskriptor. Denne Deskriptor har til opgave at beskrive punktet udfra en bestemt information, så punkterne kan sammenlignes i flere billeder. (\textit{3}) Hver deskriptor sammenlignes med andre fundne interessepunkter i andre billeder for at "matche" punkterne.
\section{Problemformulering} \label{subsec:form}
Med udgangspunkt i litteraturen inden for
korrespondanceanalyse samt implementering af
flere eksisterende metoder, hvilke metoder, teoretisk og praktisk, anvendes bedst til korrespondanceanalyse af markbilleder? \\ \\
\textbf{Udvidelse af problemformuleringen} \\
Der opstilles en beskrivelse af et udsnit af forskellige udvalgte metoder, der indgår i korrespondanceanalysens pipeline. Der vil foretages korrespondanceanalyse af markbilleder, ved implementering af udvalgte metoder. \\
Det eksperimentelle fokus i opgaven vil ligge på afprøvning af de forskellige metoder på markbilleder. Markbillederne har få distinktive træk, hvilket er egenskaber, der er vigtigt for korrespondanceanalyse. Udvælgelsen af metoder er baseret på hypoteser ift. hvordan metoderne forventes at reagere på markbillederne. Disse hypoteser vil eksperimentalt be - eller afkræftes, og derved give en empirisk forståelse for hvilke metoder, der gør korrespondance analysen mulig. Der ønskes derfor at komme frem til en metode, der kan etablere korrespondancer imellem to markbilleder og en analyse af, hvilke metoder, der gør det bedst.
\\ \\
Markbilledernes manglende diversitet, samt inkonsistente forhold under fotograferingen, diskuteret i afsnit \ref{sec:mark}, præsenterer nogle potentielle udfordringer ved korrespondanceanalysen. Udfordringerne gør det derfor interessant at undersøge, hvorvidt det er muligt at finde tilpas nok punkter, der kan anvendes. Hovedproblemet ved opgaven fokusere derfor på, hvordan der bedst kan detekteres features i markbillederne.
\subsection{Afgrænsning} \label{subsec:afg}
Projektet sigter på at afprøve allerede eksisterende metoder i et specifikt domæne og ikke
skabe nye metoder. Programmellet konstrueres mht. afklaring af de nævnte problemstillinger
og ikke mhp. efterfølgende at blive anvendt i praksis. De udvalgte metoder implementeres mhp. funktionaliteten, visse implementerings detaljer vil derfor undlades, hvis det enelige formål er at simplificere kompleksiteten.