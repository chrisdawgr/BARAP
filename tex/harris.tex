\section{Harris}\label{sec:harris}
Harris hjørnedetektor blev introduceret af Chris Harris og Mike Stephens i 1988 \cite{harris}. Metoden vil i denne opgave refereres til som Harris hjørnedetektor. Harris hjørnedetektor kan ses som en videreudvikling af Moravec, da den adressere Moravecs begrænsninger:
\begin{enumerate}
\item{ \textit{Anisotropisk respons:} Harris og Stephens udvider Moravecs udregning af auto-korrelation til at måle intensitetsvariationer i alle retninger. Auto-korrelationen imellem skiftende vinduer kan approksimeres ved at udvide funktionen, ved en \textit{"Taylor udvidelse"} af anden orden, og beskrives derved udefra dens afledte:
\begin{subequations}
\begin{align}
E(u,v) = & \sum w(x,y)[I(x + u, y + v) - I(x,y)]^2 \\
\approx & \sum w(x,y)[I(x,y) + uI_x  + vI_y - I(x,y)]^2 \\
= & \sum w(x,y)[u^2I_x^2 + 2uvI_xI_y + v^2I_y^2]  \label{last}
\end{align}
\end{subequations}
Ligning \eqref{last} kan omskrives til matrixform
\begin{equation}
E(u,v) \approx
\begin{bmatrix}
        u & v
     \end{bmatrix}
M
\begin{bmatrix}
        u \\
        v
     \end{bmatrix}
\end{equation} 
Hvor M er auto-korrelations matricen (også kaldt struktur-tensoren), med $w(x,y)$ ganget på:
\begin{equation}
M = w(x,y) 
\begin{bmatrix}
	I_x^2 & I_xI_y \\
	I_xI_y & I_y^2
\end{bmatrix}
\label{structens}
\end{equation}
Billedets afledte i x og y aksen findes ved at udføre en diskret differentiering af billedet, hvilket kan opnås
 ved at folde billedet, med henholdsvis $[-1 \hspace{0.1cm} 0 \hspace{0.1cm} 1]$ og $[-1 \hspace{0.1cm} 0 \hspace{0.1cm} 1]^T$. }
\item{\textit{Støj sensitiv:} Harris detektoren anvender et Gaussisk vindue $w(x,y)$ omkring det undersøgte pixel område. Dette tilføjer en vægt i form af en cirkel omkring punktet. Det Gaussiske vindue anvendes til at fjerne støj fra billedet, da differentiering er følsom overfor dette.}
\item{\textit{Højt kant respons} Struktur-tensoren 'M' opskrevet i \eqref{M} indeholder de differentieret retninger i et givet punkt og derved en beskrivelse af billedets geometriske struktur, i et punkt $(x,y)$. Egenværdierne for denne matrix, vil være proportionale med de forskellige steder vinduet kan placeres:
\begin{itemize}
\item{ \textit{Homogen flade.} Billedintensiteten er konstant. Begge egenværdier vil være små og der vil ikke være nogen krumning i billedet.}
\item{\textit{Kant.} For punkter placeret på en kant, vil der være en stor krumning på tværs af kanten, men ingen når kanten følges. Derfor vil én egenværdi være stor, mens den anden vil være lille}
\item{\textit{Hjørne.} Et punkt placeret på et hjørne, vil have en stor krumning, i hver retning og derved bestå af to store egenværdier.}
\end{itemize}
Det ønskes derfor, for at identificere hjørner, at begge egenværdier for M er store. Egenværdierne kan defineres ud fra determinanten og sporet af M:
\begin{subequations}
\begin{align}
\textbf{det}M & = \lambda_1 \lambda_2 \\
\textbf{trace}M & = \lambda_1+\lambda_2
\end{align}
\end{subequations}
Harris og Stephens foreslår et mål for hjørnestyrken, baseret på forholdet imellem de to egenværdier. En høj værdi for hjørnestyrken angiver to store egenværdier. Forholdet mellem egenværdierne kan defineres således:
\begin{equation}
R = \textbf{det}M-k(\textbf{trace}M)^2
\label{rvalharris}
\end{equation}
hvor \textit{k} er en empirisk defineret konstant mellem $0.04-0.06$. I figur \ref{fig:egen} ses sammenhængen mellem værdien $R$, egenværdierne og hvordan de korrespondere med hjørner, kanter og homogene områder, igen kan det ses at for et hjørne ønskes to store egenværdier.
\begin{figure}[H]
    \centering
    \includegraphics[width=0.45\textwidth]{fig/26.png}
     \vspace{-1em}
    \begin{center}    
       \caption{\textcolor{gray}{\footnotesize \textit{ Opdeling af egenværdi rummet til specifikke regioner. }}}
    \label{fig:egen}
     \end{center}
     \vspace{-2.5em}
  \end{figure} \noindent
Ved værdien for R, kan det defineres om et punkt er lokaliseret på et hjørne. De interessante hjørne, er hvor der opstår en stor "hjørnestyrke", hvilket kan findes ved at sætte en grænseværdi $t$, der definere om et hjørne er lokaliseret(sandt/falsk).
\begin{equation}
\begin{split}
\texttt{hjørne} = 
\begin{cases}
\texttt{sandt}& \texttt{hvis } R\geq t, \\
\texttt{falsk }& \texttt{hvis } R < t.
\end{cases}
\end{split}
\label{cornerhar}
\end{equation}
}
\end{enumerate}
\subsection{Algoritme}
\begin{enumerate}
\item{ For hvert pixel i billedet $(x,y)$, udregn struktur tensoren, beskrevet i ligning \eqref{structens}, hvor $ I_x^2 = (\dfrac{\partial I}{\partial x})^2 \ast w$, og $w$ er et Gaussisk vindue.}
\item{ Konstruer et billede bestående af hjørnestyrken 'R' udregnet ved ligning \eqref{rvalharris}}
\item{ Opstil en grænseværdi for billedet bestående af 'hjørnestyrken', som i ligning \eqref{cornerhar}}
\end{enumerate}
\subsection{Konklusion}
Metoden anvendes som en repeterbar detektor og bruges i høj grad i dag. Detektoren er rotationsinvariant pga. dens brug af egenværdier, men er ifølge \cite{eval} stadig delvis anisotropisk, hvilket skyldes differentierings filtrene, der kun udledes i $x$ og $y$ retningen. Mikolajczyk og Schmid \cite{eval} konkludere at detektoren er særlig følsom over for rotation omkring 45 grader. De konkludere også at detektorens repeterbarhed falder drastisk ved skalaændringer.