\subsection{Determinant of Hessian}
Determinant of Hessian $DoH$ udgør feature detektoren i SURF. $DoH$ er baseret på Hessian matricen, som udregnes for hvert punkt $p$ i et billede:
\begin{equation}
\mathcal{H}(p, \sigma) = 
 \begin{bmatrix}
 	L_{xx}(p, \sigma) & L_{xy}(p, \sigma) \\
 	L_{xy}(p, \sigma) & L_{yy}(p, \sigma) 
 \end{bmatrix}
 \label{hessianmatrix}
\end{equation}
hvor $\sigma$ angiver skala og $L_{xx}$ defineres som: 
\begin{equation}
L_{xx}(x,y, \sigma) = (\frac{\partial^2 }{\partial x^2 } G(x,y,\sigma)) * I
\label{lxx}
\end{equation}
og analogt for $L_{yy}, L_{xy}$. Bay et al. anvender approkismerede boksfiltre: $D_{xx}$, $D_{yy}$ og $D_{xy}$, der udelukkende består af værdierne $\lbrace-2,-1,0, 1\rbrace$ som ved brug af integralbilleder nedsætter antallet af beregninger drastisk. I denne implementering er disse boksfiltre ikke anvendt, men differentierede Gauss filtre er anvendt i stedet. Figur \ref{fig:lxxlyylxy} er en illustration af de afledte Gaussiske filtre, der er anvendt. Denne designbeslutning skyldes, at beregningerne af integralbilleder komplicerede programmet. Da 
boksfiltrene er en approksimation til de differentierede Gaussfiltre, der er anvendt, er denne tilgang korrekt.
\begin{figure}[H]
    \centering
    \includegraphics[width=0.75\textwidth]{fig/31.png}
     \vspace{-0.5em}
    \begin{center}    
       \caption{{\footnotesize \textit{De anvendte Gaussiske filtre, partielt differentieret i $xy,x^2$ og $y^2$.}}}
    \label{fig:lxxlyylxy}
     \end{center}
     \vspace{-2.5em}
  \end{figure} \noindent
I SURF oprettes et skalarum, der ligesom i SIFT opdeles i skalaer og oktaver. I stedet for at reducere billedernes størrelser som i SIFT, forøges størrelserne af de anvendte filtre. For hver oktav anvendes der, i denne implementering, fire billeder, foldet med fire forskellige filterstørrelser. For hver oktav bruges nr. 2 filterstørrelse, fra forrige oktav som startoktav (startende på størelse 9), og størrelsen imellem filtre, fordobles fra forrige oktav, som vist i tabel \ref{fig:secderivfiltersize}. Her skal det bemærkes, at de andenafledte filtre, skal have dobbelt størrelse, af de førsteafledte.
\begin{figure}[H]
    \centering
    \begin{center}    
    \begin{tabular}{ | l | l | l | l | l |}
    \hline
    oktav & filter str 1 & filter str 2 & filter str 3 & filter str 4 \\ \hline
    1 & 9 & 15 & 21 & 27 \\ \hline
  	2 & 15 & 27 & 39 & 51 \\ \hline
  	3 & 27 & 51 & 75 & 99 \\ \hline
  	4 & 51 & 99 & 147 & 195 \\ \hline
    \end{tabular}       
    \caption{{\footnotesize \textit{Fire forskellige oktaver, og filterstørrelse, for et andenafledt filter}}}
    \label{fig:secderivfiltersize}
     \end{center}
     \vspace{-2.5em}
  \end{figure} \noindent
For hvert punkt i billederne opstilles Hessian matricen, hvorefter determinanten udregnes for alle punkter i billederne. Bay et.al. udregner determinanten ved:
\begin{equation}
\textbf{det}\mathcal{H}_{approksimeret} = D_{xx}D_{yy}-wD_{xy}^2
\label{deerminantofhessian}
\end{equation}
hvor $w$ er en vægt, tilføjet for at balancere de approksimerede, differentierede Gaussiske filtre og boksfiltrene. Da denne implementering ikke anvender boksfiltre, udregnes determinanten som:
\begin{equation}
\textbf{det}\mathcal{H} = L_{xx}L_{yy}-L_{xy}^2
\label{deerminantofhessian}
\end{equation}
Når hessian matricens egenværdier har samme fortegn, indikere dette et ekstrema:
\begin{equation}
\begin{split}
\text{indikator} = 
\begin{cases}
\text{Ekstrema}& \text{hvis } \bold{det}(\mathcal{H}) > 0,  \\
\text{Saddel-Punkt} & \text{hvis } \bold{det}(\mathcal{H}) < 0.
\end{cases}
\end{split}
\label{detman}
\end{equation}
Et punkt udvælges derfor, hvis det er det største af et $3\times3\times3$ område af determinant billederne, som vist i figur \ref{fig:1difference}. Dette udføres for alle oktaver. Non-maximal suppression anvendes ligesom i SIFT.
\subsubsection*{Algoritme: SURF Detektor}
\begin{tabbing}
Input\quad \= : \= Billede $I$\\
Output \text{ } \> : \> Interessepunkter $p \in (x,y)$
\end{tabbing}
\begin{enumerate}
\item {Hessian matricen opstilles for alle punkter i skalabillederne som i ligning \eqref{hessianmatrix}}
\item Determinantbilleder opstilles, ved at udregne determinanten af Hessian matricen for alle punkter i alle billeder som i ligning \eqref{deerminantofhessian}.
\item Lokale maxima udvælges, af hvert $3\times3\times3$ område af determinantbillederne på samme oktav.
\item Non-maximal suppression, bruges til at fjerne dårligt lokaliserede punkter ligesom i ligning \eqref{maxsurp}.
\end{enumerate}