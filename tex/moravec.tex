\section{Moravec}\label{sec:moravec}
%Moravecs hjørnedetektor\cite{moravec}, er en af de tidligste feature detektorer og i denne metode defineres et hjørne, som et dataindsamlingsvindue i et billede $I$, der er forskelligt fra dets lokalområde. <?>
Moravecs hjørnedetektor\cite{moravec}, er en af de tidligste feature detektorer. Moravec definere et hjørne ved at placere et kvadratisk dataindsamlingsvindue over et punkt, hvorefter dette vindue forskydes med én pixel i alle principielle retninger (horisontalt, vertikalt, diagonalt). Forskellen imellem det originale og de forskudte vinduer observeres, og en stor forskel imellem vinduerne definere et hjørne. Moravec definere altså et hjørne som værende et område, hvori der forekommer store ændringer. Moravec definere forskellen imellem de forskudte vinduer ved at estimere auto-korrelationen imellem det originale og de forskudte vindue. Denne auto-korrelation, eller forksel intensitetvariation, estimeres ved SSD (Summed Square Difference):
\begin{equation}
\bold{E}(u,v)= \sum_{\forall \text{x,y i dataindsamlingsvinduet}} w(x,y)[I(x+u,y+v) - I(x,y)]^2
\label{moravec}     
\end{equation}
hvor $(u,v)\in \lbrace -1,0,1 \rbrace$ er forskydningen af vinduet, der summere over alle pixels i dataindsamlingsvinduet $w$. 
%<$w$ er her en matrix<?>, med samme størrelse som $I$ <??> og indeholder 1 indenfor hvis en pixelkoordinat $(x,y)$ ligger indenfor dataindsamlingsvinduet, og 0 ellers<?>.>
Figur \ref{fig:moravec} illustrerer et dataindsamlingsvindue af størrelse 3x3 (markeret med rød). Det blå vindue viser et diagonalt skift af dataindsamlingsvindue med $(u,v)=(1,1)$. Det ses, at det røde vindue til højre, ligger på et ideelt hjørne. 
\begin{figure}[H]
    \centering
    \includegraphics[width=0.55\textwidth]{fig/25.png}
     \vspace{-1em}
    \begin{center}    
       \caption{\textcolor{gray}{\footnotesize \textit{ SSD udregninger for intensitetsvariationer imellem forskudte vindue. }}}
    \label{fig:moravec}
     \end{center}
     \vspace{-2.5em}
  \end{figure} \noindent   
$E(u,v)$ beregnes otte gange for hver pixel. Den mindste forskel af de otte variationer, definere punktets hjørnestyrke $C(x,y)$.
$$
C(x,y)=min(E(u,v)) \forall u,v \in [-1,0,1] \backslash \lbrace0,0\rbrace
$$
En horisontal kant vil resultere i en lav intensitetsvariation med et dataindsamlingsvindue forskudt horisontalt derfor, for at identificere hjørner, tages minimum af $E$ for at sikre en stor auto-korrelation i alle retninger.
%<vil have et højt respons i de områder, hvor pixelintensiteter i originalvinduet ($w(x,y) = 1$) <?>, adskiller sig fra intensiteter, i de forskudte vinduer - jo større pixelintensitens forskellen er, jo højere $C$ værdi.
Det kan udledes her, at en kant, med retning i en de principielle retninger, ikke vil blive karakteriseret som et hjørne da disse vil have  en lav auto-korrelation.
Der opstilles en grænseværdi $t$ for $C$, der bestemmer om punktet er et hjørne(sandt/falsk) og derved om det kan udvælges som et interessepunkt.
\begin{equation}
\begin{split}
\text{hjørne} = 
\begin{cases}
\text{sandt}& \text{hvis } C(x,y)\geq t, \\
\text{falsk }& \text{hvis } C(x,y) < t.
\end{cases}
\end{split}
\label{cornerind}
\end{equation}
Moravec lider af følgende problemer pga. dens simplicitet:
\begin{itemize}
\item{Der undersøges kun et diskret sæt af pixelskift (i hver principiel retning) og resultatet er derfor anisotropisk, (afhængig af retning). Undersøges en kant, der ikke er horisontal, vertikal eller diagonal, vil den mindste intensitetsvariation være stor, og derved kan punktet fejlagtigt detekteres som et hjørne.}
\item{Det skiftende vindue er rektangulært, og metoden er derfor meget følsom overfor støj i billedet.}
\item{Detektoren finder punkter lokaliseret på kanter. Små deformationer i kanterne som støj, vil resultere i at den mindste intensitetsvariation vil være relativt stor, og derfor detektere punktet som værende et interessepunkt.}
\end{itemize}
% <evt konklusion>
% <måske billeder download cornerdetection.pdf>
\subsection{Algoritme}
\begin{enumerate}
\item{For hvert pixel i billedet, udregn auto-korrelationen imellem skift af $(u,v) \in \lbrace-1,0,1\rbrace$. udregnet ved ligning \ref{moravec}}
\item{\textit{Find "Hjørnestyrken"} ved at tage $C(x,y)=min(E(u,v))$}
\item{Fjern punkter der ikke overholder en sat grænseværdi for $C(x,y)$.}
\end{enumerate}
\subsection{Konklusion}
Moravec er som nævnt en simpel algoritme, med mange udfordringer, der gør at den ikke bruges som en repeterbar detektor. Detektoren er i dag ikke i sig selv relevant, som den var da den blev udgivet, men bygges videre på i andre detektorer, f.eks. \cite{Harris} beskrevet i sektion, \ref{sec:harris} som direkte tilgår de nævne problemstillinger med Moravec.