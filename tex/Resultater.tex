\chapter{Resultater}
\label{sec:resultater}
Denne sektion har til formål at præsentere opgavens resultater. For at evaluere metoderne, er en måling for metodernes repeterbarhed opstillet, der udtrykker forholdet imellem korrekte korrespondancer og middelværdien af det samlede antal interessepunkter fundet i begge billeder \cite{eval}:
\begin{equation}
r_{I_1,I_2)}=\dfrac{C(I_1,I_2)}{\text{mean}(n_1,n_2)}
\end{equation}
hvor $C(I_1,I_2)$ angiver antallet af korrekte korrespondancer imellem to billeder og $n_1=\bold{|A|}$ og $n_2=\bold{|A'|}$.
%$\text{mean}(\bold{|P_1|},\bold{|P_2|)}$ angiver middelværdien af det samlede antal af fundne interessepunkter i det to billeder
%<beskriv hvordan punkter fjernes>
Moravec og Harris er blevet modificeret, for at få dem til at passe med SIFT. I begge tilfælde foldes billederne med en Gausskerne, og nedskaleres efter $\sigma$, som illustreret figur <???>, hvorefter detektionsmetoderne anvendes. 
\\
\\
I nedenstående tabeller angiver Filename $x$ filnavnene på de billeder der er fundet korrespondancer mellem. $|A|$ og $|A'|$ er defineret som i afsnit 3.5. $Match(A, A')$ er antallet af fundne korresponderende punkter, givet matching algoritmen fra afsnit 5.5. Givet det store antal matchende punkter, er det her antaget, at alle punkter fundet ved $Match$ algoritmen er sande positiver <korrekte korrespondancer> - mere om dette er omtalt i diskussionen af resultaterne. De 15 bedste matches, er også vedlagt som billede, for hver metode <de kan findes i fuld størrelse i..>
\\
\\
De resultater der her er fremlagt og senere diskuteres, er fra en enkelt skala, fra en enkelt oktav. SIFT og SURF søger oprindeligt efter blobs i flere skalrum - disse udregninger indgår ikke i resultat sektionen. 
\section{Moravec}
\section{Harris}


\begin{figure}[H]
    \centering
    \begin{center}    
    \begin{tabular}{ | l | l | l | l | l | l | l |}
    \hline
    Filename 1 & Filename 1 & |$\bold{A}$| & |$\bold{A'}$| & $mean(A,A')$ & $Match(\bold{A}, \bold{A}')$ & $Rm$ \\ \hline
IMG\_9370.jpg &	IMG\_9371.jpg &	4327 &	2616 &	3471.5 &	563 &	0.1621\\ \hline
IMG\_9371.jpg &	IMG\_9372.jpg &	2616 &	1842 &	2229.0 &	389 &	0.1745\\ \hline
IMG\_9372.jpg &	IMG\_9373.jpg &	1842 &	4376 &	3109.0 &	182 &	0.0585\\ \hline
IMG\_9373.jpg &	IMG\_9374.jpg &	4376 &	1873 &	3124.5 &	233 &	0.0745\\ \hline
IMG\_9374.jpg &	IMG\_9375.jpg &	1873 &	972 &	1422.5 &	172 &	0.1209\\ \hline
IMG\_9375.jpg &	IMG\_9376.jpg &	972 &	1429 &	1200.5 &	238 &	0.1982\\ \hline
IMG\_9376.jpg &	IMG\_9377.jpg &	1429 &	4036 &	2732.5 &	367 &	0.1343\\ \hline
IMG\_9377.jpg &	IMG\_9378.jpg &	4036 &	2760 &	3398.0 &	455 &	0.1339\\ \hline
IMG\_9378.jpg &	IMG\_9379.jpg &	2760 &	688 &	1724.0 &	376 &	0.2180\\ \hline
IMG\_9379.jpg &	IMG\_9380.jpg &	688 &	1154 &	921.0  &	212 &	0.2301\\ \hline
    \end{tabular}       
    \caption{\textcolor{gray}{\footnotesize \textit{Fire forskellige oktaver, og filterstørrelse, for et andenafledt filter}}}
    \label{tab:HARRISOCTAVE2}
     \end{center}
     \vspace{-2.5em}
\end{figure} \noindent
Middelværdi af repeatability mesaure  = 0.1505


\begin{figure}[H]
    \centering
    \includegraphics[width=1.11\textwidth]{fig/10best-HarrisSift-octave1.jpg}
    \vspace{-0.5em}   
    \begin{center}
    \caption{\textcolor{gray}{\footnotesize \textit{
    Tre lyse blobs på en mørk baggrund illustreret i 2 dimensioner \cite{blob}}}}
    \label{fig:lindblob}
     \end{center}
  \end{figure}
       \vspace{-2.7em}
\noindent





\section{SIFT}
\begin{figure}[H]
    \centering
    \begin{center}    
    \begin{tabular}{ | l | l | l | l | l | l | l |}
    \hline
    Filename 1 & Filename 1 & |$\bold{A}$| & |$\bold{A'}$| & $mean(A,A')$ & $Match(\bold{A}, \bold{A}')$ & $Rm$ \\ \hline
IMG\_9370.jpg &	IMG\_9371.jpg &	10914 &	13590 &	12252.0 &	237 &	0.0193\\ \hline
IMG\_9371.jpg &	IMG\_9372.jpg &	13590 &	14457 &	14023.5 &	351 &	0.0250\\ \hline
IMG\_9372.jpg &	IMG\_9373.jpg &	14457 &	10146 &	12301.5 &	152 &	0.0123\\ \hline
IMG\_9373.jpg &	IMG\_9374.jpg &	10146 &	9625 &	9885.5 &	100 &	0.0101\\ \hline
IMG\_9374.jpg &	IMG\_9375.jpg &	9625 &	10288 &	9956.5 &	184 &	0.0184\\ \hline
IMG\_9375.jpg &	IMG\_9376.jpg &	10288 &	11735 &	11011.5 &	266 &	0.0241\\ \hline
IMG\_9376.jpg &	IMG\_9377.jpg &	11735 &	10455 &	11095.0 &	345 &	0.0310\\ \hline
IMG\_9377.jpg &	IMG\_9378.jpg &	10455 &	9026 &	9740.5 &	271 &	0.0278\\ \hline
IMG\_9378.jpg &	IMG\_9379.jpg &	9026 &	9095 &	9060.5 &	567 &	0.0625\\ \hline
IMG\_9379.jpg &	IMG\_9380.jpg &	9095 &	7545 &	8320.0 &	509 &	0.0611\\ \hline
    \end{tabular}       
    \caption{\textcolor{gray}{\footnotesize \textit{Fire forskellige oktaver, og filterstørrelse, for et andenafledt filter}}}
    \label{tab:SURFOCTAVE2}
     \end{center}
     \vspace{-2.5em}
\end{figure} \noindent
Middelværdi af repeatability mesaure = 0.0292

\begin{figure}[H]
    \centering
    \includegraphics[width=1.11\textwidth]{fig/10best-SIFT-octave2.jpg}
    \vspace{-0.5em}   
    \begin{center}
    \caption{\textcolor{gray}{\footnotesize \textit{
    Tre lyse blobs på en mørk baggrund illustreret i 2 dimensioner \cite{blob}}}}
    \label{fig:lindblob}
     \end{center}
  \end{figure}
       \vspace{-2.7em}
\noindent





\section{SURF}
\begin{figure}[H]
    \centering
    \begin{center}    
    \begin{tabular}{ | l | l | l | l | l | l | l |}
    \hline
    Filename 1 & Filename 1 & |$\bold{A}$| & |$\bold{A'}$| & $mean(A,A')$ & $Match(\bold{A}, \bold{A}')$ & $Rm$ \\ \hline
IMG\_9370.jpg &	IMG\_9371.jpg &	12529 &	12882 &	12705.5 &	3249 &	0.2557\\ \hline
IMG\_9371.jpg &	IMG\_9372.jpg &	12882 &	14700 &	13791.0 &	3422 &	0.2481\\ \hline
IMG\_9372.jpg &	IMG\_9373.jpg &	14700 &	14636 &	14668.0 &	557 &	0.0379\\ \hline
IMG\_9373.jpg &	IMG\_9374.jpg &	14636 &	14352 &	14494.0 &	622 &	0.0429\\ \hline
IMG\_9374.jpg &	IMG\_9375.jpg &	14352 &	13902 &	14127.0 &	566 &	0.0400\\ \hline
IMG\_9375.jpg &	IMG\_9376.jpg &	13902 &	12200 &	13051.0 &	2575 &	0.1973\\ \hline
IMG\_9376.jpg &	IMG\_9377.jpg &	12200 &	11614 &	11907.0 &	2800 &	0.2351\\ \hline
IMG\_9377.jpg &	IMG\_9378.jpg &	11614 &	10028 &	10821.0 &	1005 &	0.0928\\ \hline
IMG\_9378.jpg &	IMG\_9379.jpg &	10028 &	10600 &	10314.0 &	1911 &	0.1852\\ \hline
IMG\_9379.jpg &	IMG\_9380.jpg &	10600 &	9955 &	10277.5 &	1761 &	0.1713\\ \hline
    \end{tabular}       
    \caption{\textcolor{gray}{\footnotesize \textit{Fire forskellige oktaver, og filterstørrelse, for et andenafledt filter}}}
    \label{tab:SIFTOCTAVE3}
     \end{center}
     \vspace{-2.5em}
\end{figure} \noindent
Middelværdi af repeatability mesaure = 0.2342


\begin{figure}[H]
    \centering
    \includegraphics[width=1.11\textwidth]{fig/10best-SURF-octave3.jpg}
    \vspace{-0.5em}   
    \begin{center}
    \caption{\textcolor{gray}{\footnotesize \textit{
    Tre lyse blobs på en mørk baggrund illustreret i 2 dimensioner \cite{blob}}}}
    \label{fig:lindblob}
     \end{center}
  \end{figure}
       \vspace{-2.7em}
\noindent



\section{Diskussion}


Denne beslutning er taget på baggrunden af rationalet, at markbilleder taget af en drone fra en bestemt kendt højde, skaber blobs af en bestemt størrelse. Det er derfor ikke nødvendigt at kigge efter blobs på andre skaler, end den, der empirisk har vist sig at give de bedste resultater  ved en forudgående undersøgelse.


\section{Forbedringer}
Følgende afsnit vil gennemgå hvilke forbedringer de implementerede metoder kan gennemgå, for at forbedre korrespondanceanalysen.
\subsection{Non-maximal suppression}
Som beskrevet i afsnit \ref{sec:dog}, og som gør sig gældende for både Difference of Gaussian og Determinant of Hessian metoderne,  er non-maximal suppression ikke implementeret for at finde sub-pixel nøjagtighed. 
\\
\\
I denne implementering bliver et punkt kun valgt, hvis $|\hat{x}| < 0.5$, i en given retning. $\hat{x}$ angiver en retningsvektor i tre dimensioner$(x, y, \sigma)$, der er orienteret mod et ekstrema og bruges i denne implementering til at frasortere punkter, der ikke er placeret på ekstremaer. Punkters placering er ikke blevet fundet ned til sub-pixel nøjagtighed. Det blev ikke anset som en nødvendighed i denne udgave, da første prioritet var at finde korrespondancer og illustrere dem - her er sub-pixel nøjagtighed ikke brugbart, da hver billedkoordinat er et naturligt heltal. I en praktisk anvendelse, kan det ønskes, at der skal etableres homografier, eller bruge andre billedanalytiske metoder, på de fundne korrespondancer. Her er sub-pixel nøjagtighed strengt nødvendigt, for at få nøjagtige resultater. Implementering af non-maximal suppression som forklaret af Lowe, er derfor næste logiske udvidelse af programmet. 

\subsection{SIFT}
I figur \ref{fig:graf}, ses det, at den udregnede repeatability measure for SIFT er langt lavere, end de andre metoder. Det er lykkedes at finde korrespondancer med den samlede SIFT metode, der er dog langt færre matches, end i Harris/SIFT og SURF. Dette resultat er i uoverensstemmelse med andre sammenligninger af metoder\cite{kim} \cite{kim2}. I \cite{kim} sammenligner Pedersen et al. forskellige metoder med billeder udsat for forskellige ændringer, under kontrollerede forhold. En recall rate, der svarer til den repetability measure brugt her til at evaluere metoderne. Pedersen et al. opsætter nogle strengere krav(mere specifikt, tre krav) for, hvornår to punkter korrespondere ($C(I_1, I_2)$), end der gøres i denne opgave. Kim et al. konkludere at Harris og Difference of Gaussian giver bedere resultater end Determinant of Hessian metoden. Denne konklusion er i modstrid med resultaterne præsenteret her. Selvom applikationsområderne er forskellige, forventedes der alligevel en vis lighed imellem resultaterne. Det må derfor, på baggrund af den udledte repeatability measure for SIFT, konkluderes at der er fejl i implementeringen af SIFT.
\subsection{RGB til gråtone}
Som nævnt i afsnit 3.1, blev der anvendt en specifik metode, til at transformere intensiteter i RGB til gråtone værdier. Det kunne være interessant at undersøge, hvorvidt Excess Green(ExG) kan bruges til at øge repeatability measure. ExG kan bruges til at adskille planter fra jord\cite{exg}. Dette kunne skabe større kontrast i billedet og derfor muligvis tydeliggøre blobs og hjørner.
