
\subsection{SURF}
SURF, ligesom SIFT, er en fuldstændig løsning, til at finde og beskrive blobs, som består af en detektor og en deskriptor. I forhold til SIFT, har SURF den fordel, at den beregningsmæssigt kan optimeres til at være hurtigere end SIFT, og metoden er mindre kompleks, fordi den består af færre beregningsskridt. SURF kan optimeres, igennem brug af integralbilleder[???].
\subsection*{Detektor}
SURF detektoren, er baseret på en Determinant of Hessian (DoH) matrix.
\\
\\
Hessian matricen, ser ud på følgende måde:
\begin{equation}
\mathcal{H}(p, \sigma) = 
 \begin{pmatrix}
 	L_{xx}(p, \sigma) & L_{xy}(p, \sigma) \\
 	L_{xy}(p, \sigma) & L_{yy}(p, \sigma) 
 \end{pmatrix}
 \label{hessianmatrix}
\end{equation}
hvor $p = (x,y)$, og $\sigma$ svarer til skalaen, og 
\begin{equation}
L_{xx}(x, \sigma) = (\frac{\partial^2 }{\partial x^2 } G(x,y,\sigma)) * I
\end{equation}
$L_{xy}$ og $L_{yy}$ er Gaussfunktionen partielt differentieret ifht. $xy$ og $yy$.
\\
Ovenstående kan udregnes, ved at bruge en diskret approksimation, til den differentieret Gauss, og folde denne over billedet $I$. Boxfiltrene brugt, kan ses på figur <??>. For at få repræsentere forskellige skaler, bliver der brugt først brugt Gaussiske boxfilter, med størrelse 9x9, og $\sigma = 1.2$. I SURF bliver $I$ ikke subsamplet, men størrelsen på filtret der foldes over billedet, bliver større. For her oktav, eksisterer (i denne implementering) fire billeder, foldet med fem forskellige størrelse filterstørrelser. For hver skala, bruges nr. to filterstørrelse, fra forrige skala, og størrelsen mellem filtre, fordobles fra forrige skala, startende med filterstørrelse 9, og afstand mellem filtre 6. Dette er vist tabel \ref{fig:secderivfiltersize}. Her skal det bemærkes, at de andenafledte filtre, skal have dobbelt størrelse, af de førsteafledte, som er vist i tabel \ref{fig:firstderivfiltersize}


\begin{figure}[H]
    \centering
    \begin{center}    
    \begin{tabular}{ | l | l | l | l | l |}
    \hline
    oktav & filter str 1 & filter str 2 & filter str 3 & filter str 4 \\ \hline
    1 & 9 & 15 & 21 & 27 \\ \hline
  	2 & 15 & 27 & 39 & 51 \\ \hline
  	3 & 27 & 51 & 75 & 99 \\ \hline
  	4 & 51 & 99 & 147 & 195 \\ \hline
    \end{tabular}       
    \caption{\textcolor{gray}{\footnotesize \textit{Fire forskellige oktaver, og filterstørrelse, for et andenafledt filter}}}
    \label{fig:secderivfiltersize}
     \end{center}
     \vspace{-2.5em}
  \end{figure} \noindent

\begin{figure}[H]
    \centering
    \begin{center}    
    \begin{tabular}{ | l | l | l | l | l |}
    \hline
    oktav & filter str 1 & filter str 2 & filter str 3 & filter str 4 \\ \hline
    1 & 3 & 5 & 7 & 9 \\ \hline
  	2 & 5 & 9 & 13 & 17 \\ \hline
  	3 & 9 & 17 & 25 & 33 \\ \hline
  	4 & 17 & 33 & 49 & 65 \\ \hline
    \end{tabular}       
    \caption{\textcolor{gray}{\footnotesize \textit{Fire forskellige oktaver, og filterstørrelse, for et førsteafledt filter}}}
    \label{fig:firstderivfiltersize}
     \end{center}
     \vspace{-2.5em}
  \end{figure} \noindent
Determinanten af \eqref{hessianmatrix} udregnes nu, for alle billederne:
\begin{equation}
\textbf{det}(\mathcal{H}_{approksimeret} = D_{xx}D_{yy}-0.9D_{xy}^2
\label{deerminantofhessian}
\end{equation}
Herefter bliver der, ligesom i SIFT, fundet et maximum, af et 3x3x3 området af $I$, som vist \eqref{fig:difference}. Dette gøres for alle oktaver. Billederne gennemgår herefter non-maximum suppression, som også beskrevet i SIFT afsnittet.

\subsection*{Algoritme}
\begin{enumerate}
\item Hessian matricen udregnes for alle oktaver, som vist i ligning \eqref{hessianmatrix}
\item Determinanten findes, som \eqref{deerminantofhessian}.
\item Maximum findes, af hvert 3x3x3 område af billeder på samme oktav
\item Non-maximum suppression bruges til at finde extremaer.
\end{enumerate}





\subsection*{Orientering}


\begin{figure}[H]
    \centering
    \includegraphics[width=0.65\textwidth]{fig/haarwavelet.png}
     \vspace{-1em}
    \begin{center}    
       \caption{\textcolor{gray}{\footnotesize \textit{ }}}
    \label{fig:haarwavelet}
     \end{center}
     \vspace{-2.5em}
  \end{figure} \noindent

SURF gør brug af Haar-wavelet, både i detektoren og deskriptoren. Haar-wavelet er et boks-filter, med størrelse nxn, hvor halvdelen af alle indgangene er +1 og den anden halvdel er -1.


\subsubsection{Deskriptor}
SURF deskriptoren producerer features, der består af 64 indgange. Den kan beskrives stringent, ligesom (26). Som beskrevet i SURF papiret af Tuytelaars et. al[??], er tilfældet ofte, at der ikke er brug for rotationsinvarians. En variant af SURF, der ikke er rotationsinvariant, kaldes Upright-SURF (U-SURF). Tuytelaars et. al garanterer dog rotationsinvarians i U-SURF, i op til $\pm 15^{\circ}$. SURF er implementeret, fremfor U-SURF, da enkelte af billederne har stor grad (ca. $15^{\circ}$) af rotation. Rotationen forekommer i billeder, der er taget lige efter, at dronen har vendt og kan ses på figur 15.
\\
\\
For at gøre SURF invariant overfor rotation, skal Haar-wavelet responset findes i x og y retningen omkring punktet ($dx, dy$), respektivt. $dx$ og $dy$ beregnes i en cirkel omkring interessepunktet, med radius $6\sigma$, og en afstand mellem punkterne, på $\sigma$. $dx$ og $dy$ foldes med en cirkulær Gausskerne, med samme størrelse som Haar-wavelet responsene, og $\sigma_{Gauss} = 2.5\sigma$.
\begin{figure}[H]
    \centering
    \includegraphics[width=0.2\textwidth]{fig/surforientation.jpg}
     \vspace{-1em}
    \begin{center}    
       \caption{\textcolor{gray}{\footnotesize \textit{ }}}
    \label{fig:surforientation}
     \end{center}
     \vspace{-2.5em}
  \end{figure} \noindent
Et sliding vindue på $60^{\circ}$, summerer alle vektorene, der ligger indenfor dets rækkevidde. Den længste vektor, udgør retningen, som beregnes med $atan2$ (invers tangens, med 2 inputs). Dette er illustreret på figur \ref{fig:surforientation}. Der er her taget en beslutning om, at slidingvinduet udregner summerne af vektorer, indenfor seks positioner ($0^{\circ}-60^{\circ}$, $60^{\circ}-120^{\circ}$.. $300^{\circ}-360^{\circ}$). 
\\
\\
SURF deskriptoren samler data omkring interssepunktet, med et dataindsamlingvindue der har størrelse $20 \sigma$. Vinduet skal orienters, ifht. den $\theta$ værdi, der er udregnet i detektoren. Billedet vendes, ved, at vende hele billedet, ved brug af ligning \eqref{rotaionmatrix}, og danne et integralbillede udfra det nye billede (Dette skridt er omkostningsfuldt, og der vil senere blive diskuteret optimeringer).
\\
\\
Dataindsamlingsvinduet er nu roteret og centreret omkring interessepunktet. Herefter skal vinduet deles op i 4x4 regioner, hver bestående af 5x5 punkter, der har ens afstand mellem sig - her er ikke blevet lavet subpixel-optimeringen, så hvis $\frac{20\sigma}{4}$ ikke er deleligt med 5, bliver der rundet ned.
\\
\\
Herefter skal Haar-wavelet responset udreges, i x og y retningen ($dx, dy$, respektivt). Haar-wavelet skal udregnes for alle punkter i 5x5 felterne. Størrelsen på Haar-wavelet filteret, er $2\sigma$. Disse værdier skal smoothes med et Gaussfilter, hvor $\sigma_{Gauss} = 3.3\sigma_{point}$.
\\
\\
For her af de 16 4x4 regioner, udregnes: 
\begin{equation}
v_i = \sum d_x, \sum d_y, \sum |d_y|, \sum |d_y|
\label{surffeature}
\end{equation}
hvor $v_i$ er beregningerne, for den $i$'ende region. Der laves en vektor, hvor alle værdierne indgange i $v$ sættes efter hinanden. Dette giver en $16 \cdot 4 = 64$ indgange stor vektor. Slutteligt, laves vektoren om til en enhedsvektor.

\subsubsection*{Algoritme}
\begin{enumerate}
\item Orienteringen findes: \begin{enumerate}

	\item Orienteringen på punktet findes, ved at beregne $dx$, $dy$ i et cirkulært område, med radius $6\sigma$, og afstand $\sigma$, mellem punkterne. 
	\item $dx$, $dy$ foldes med en Gauss kerne, med $\sigma_{Gauss} = 2.5\sigma $
	\item Den længste vektor, af summerede vektorer, der ligger indenfor $60^{\circ}$, bruges til at tildele orientering $\theta$ til interessepunktet.
\end{enumerate}
\item Et dataindsamlingsvindue roteret ifht. $\theta$, beregnet som \eqref{rotaionmatrix} med størrelse $20\sigma$ indsamler punkter, der er har afstand $\sigma$ mellem hinanden.
\item Haar-wavelet responset, $dx$, $dy$ udregnes for de indsamlede punkter, med en filterstørrelse på $2\sigma$, og $dx, dy$ foldes med Gauss, hvor $\sigma_{Gauss} = 3.3\sigma$
\item For hver af det 4x4 regioner, udregnes \eqref{surffeature}. Dette sættes sammen til en vektor, der vil have 64 indgange
\end{enumerate}