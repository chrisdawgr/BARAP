\subsection{Skalarum}
Objekter i virkeligheden, såvel som detaljer i et billede, optræder kun som meningsfulde enheder over et specifikt skalainterval. Et simpelt eksempel er et træ, hvilket, indenfor meters afstand, vil opfattes som et træ, men indenfor centimeters eller nanometers afstand vil optræde som blade og molekyler. I et billede vil de tætteste objekter, optræde som fine strukturer og de længere væk, som grovere strukturer. 
\\
Formelt kan et billede repræsenteres som en funktion af to variable:
\begin{equation}
\begin{split}
&I: R^2 \rightarrow R \\
&I(x,y) = \lambda \hspace{0.5 cm} (x,y)\in R^2, \lambda \in R
\end{split}
\end{equation}
hvor $\lambda$ repræsentere en billedintensitet. Skalaen i et billede diktere, hvordan et objekt opfattes. Da billedet indeholder en ukendt scene, vides det ikke hvilket skalainterval der er interessant for et givent objekt. For at løse dette problem, oprettes en multi-skala repræsentation af billedet, bundet til en enkelt parameter også kaldt skalaparametren $\sigma$. Derfor kan billedfunktionen udvides til en funktion, der tager 3 parametre:
\begin{equation}
\begin{split}
&L: R^3 \rightarrow R \\
&L(x,y,\sigma) = \lambda \hspace{0.5 cm} (x,y,\sigma)\in R^3, \lambda \in R
\end{split}
\end{equation}
For at opnå en multi-skal repræsentation af billedet, skal der oprettes et skalarum bestående af skalabilleder, der går fra at fremhæve de finere strukturer til de grovere strukturer, proportionelt med skalaparametren som illustreret i figur \ref{fig:scalerep}. 
\begin{figure}[H]
    \centering
    \includegraphics[width=0.45\textwidth]{fig/32.png}
     \vspace{-1em}
    \begin{center}    
       \caption{\textcolor{gray}{\footnotesize \textit{ }}}
    \label{fig:scalerep}
     \end{center}
     \vspace{-2.5em}
  \end{figure} \noindent
Denne overgang fra finere til grovere strukturer kan opnås ved iterativt at folde billederne med et Gaussisk filter af stigende $\sigma$ værdier:
\begin{equation}
L(x,y,\sigma) = G(x,y,\sigma)\ast I(x,y)
\label{scalespace1}
\end{equation}
, hvor billedets nulskala repræsenteres ved biledet $ L(x,y,0) = I(x,y)$. \\
Der anvendes et Gaussisk filter, på grund af dens unikke egenskaber,  bl.a. at der ikke forekommer nye strukturer ved glatning \cite{witkins}, når skalaparametren stiger. Strukturer, der findes på grovere skalaer, må ikke være nyopståede objekter, men derimod simplificeringer af objekter eksisterende på de finere skalaer, hvilket tillades af det Gaussiske filter. Et skalarum repræsentation anvender en kontinuerlig stigende skalaparameter, der vedligeholder det samme skalainterval og niveau af glatning imellem skalaer.
Figur \ref{scalespace1} viser resultatet af at glatte et signal med et Gaussisk filter af stigende $\sigma$ værdi. Det ses tydeligt, hvordan signalets underliggende grovere struktur bliver fremhævet og at finere strukturer bliver fjernet.
\begin{figure}[H]
    \centering
    \includegraphics[width=0.45\textwidth]{fig/33.png}
     \vspace{-1em}
    \begin{center}    
       \caption{\textcolor{gray}{\footnotesize \textit{ }}}
    \label{fig:scalereps}
     \end{center}
     \vspace{-2.5em}
  \end{figure} \noindent
En anden, udbredt måde, at repræsentere et skalrum er ved en \textit{skalapyramide}.  