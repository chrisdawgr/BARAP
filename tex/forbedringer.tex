\section{Forbedringer}
Følgende afsnit vil gennemgå hvilke forbedringer de implementerede metoder kan gennemgå, for at forbedre korrespondanceanalysen.
\subsection{Non-maximal suppression}
Som beskrevet i afsnit \ref{sec:dog}, og som gør sig gældende for både Difference of Gaussian og Determinant of Hessian metoderne,  er non-maximal suppression ikke implementeret for at finde sub-pixel nøjagtighed. 
\\
\\
I denne implementering bliver et punkt kun valgt, hvis $|\hat{x}| < 0.5$, i en given retning. $\hat{x}$ angiver en retningsvektor i tre dimensioner$(x, y, \sigma)$, der er orienteret mod et ekstrema og bruges i denne implementering til at frasortere punkter, der ikke er placeret på ekstremaer. Punkters placering er ikke blevet fundet ned til sub-pixel nøjagtighed. Det blev ikke anset som en nødvendighed i denne udgave, da første prioritet var at finde korrespondancer og illustrere dem - her er sub-pixel nøjagtighed ikke brugbart, da hver billedkoordinat er et naturligt heltal. I en praktisk anvendelse, kan det ønskes, at der skal etableres homografier, eller bruge andre billedanalytiske metoder, på de fundne korrespondancer. Her er sub-pixel nøjagtighed strengt nødvendigt, for at få nøjagtige resultater. Implementering af non-maximal suppression som forklaret af Lowe, er derfor næste logiske udvidelse af programmet. 

\subsection{SIFT}
I figur \ref{fig:graf}, ses det, at den udregnede repeatability measure for SIFT er langt lavere, end de andre metoder. Det er lykkedes at finde korrespondancer med den samlede SIFT metode, der er dog langt færre matches, end i Harris/SIFT og SURF. Dette resultat er i uoverensstemmelse med andre sammenligninger af metoder\cite{kim} \cite{kim2}. I \cite{kim} sammenligner Pedersen et al. forskellige metoder med billeder udsat for forskellige ændringer, under kontrollerede forhold. En recall rate, der svarer til den repetability measure brugt her til at evaluere metoderne. Pedersen et al. opsætter nogle strengere krav(mere specifikt, tre krav) for, hvornår to punkter korrespondere ($C(I_1, I_2)$), end der gøres i denne opgave. Kim et al. konkludere at Harris og Difference of Gaussian giver bedere resultater end Determinant of Hessian metoden. Denne konklusion er i modstrid med resultaterne præsenteret her. Selvom applikationsområderne er forskellige, forventedes der alligevel en vis lighed imellem resultaterne. Det må derfor, på baggrund af den udledte repeatability measure for SIFT, konkluderes at der er fejl i implementeringen af SIFT.
\subsection{RGB til gråtone}
Som nævnt i afsnit 3.1, blev der anvendt en specifik metode, til at transformere intensiteter i RGB til gråtone værdier. Det kunne være interessant at undersøge, hvorvidt Excess Green(ExG) kan bruges til at øge repeatability measure. ExG kan bruges til at adskille planter fra jord\cite{exg}. Dette kunne skabe større kontrast i billedet og derfor muligvis tydeliggøre blobs og hjørner.