\section{Deskriptor}
Fra en detektor genereres et sæt interessepunkter $\bold{P}$, fra to eller flere billeder som vist i ligning \eqref{detect}. For at udvælge korresponderende punkter imellem billederne, skal interessepunkterne beskrives af en deskriptor $Des$. Deskriptoren er en funktion, der tager et billede og et punkt som input. Deskriptoren beskriver dette punkt, udefra den lokale billedstruktur, omkring og i punktet og tilegner punktet en beskrivelse, kaldet en feature $f$:
$$ Des(I,p_i)=f_i $$
$$ Des(I',p_j')=f_j' $$
For to korresponderende punkter $i$ og $j$ ønskes det at $f_i \approx f'_j$, så punkterne korrekt kan identificeres som værende korresponderende punkter, men også for to \textit{ikke} korresponderende punkter $k$ og $l$ at $f_k \not\approx f'_l$ så de kan identificeres som ikke korresponderende. Den tilegnede beskrivelse kommer i form af en n-dimensional vektor\footnote{For en given metode vil længden af disse vektorer være ens}, hvor indgangende i vektoren indeholder information om området omkring interessepunktet, hvilket kan bruges til at determinere, ligheden af interessepunkterne.
$$ f =
\begin{bmatrix}
d_1 \\
d_2 \\
. . . \\
d_n
\end{bmatrix}
$$
Generelt ønskes der at en deskriptor besidder følgende egenskaber:
\begin{itemize}
\item{\textit{Robust}: Deskriptoren skal kunne identificere to korresponderende punkter som værende ens på trods af ændringer i billedet.}
\item{\textit{Invariant}: Deskriptoren skal være invariant overfor de ændringer, der kan forekomme under fotograferingen.}
\end{itemize}

