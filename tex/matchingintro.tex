\section{Matching}
Fra deskriptoren er et sæt interessepunkter $p$ beskrevet ved features:
$$ Des(I,p)= \lbrace F_1,F_2,...,F_n \rbrace $$
$$ Des(I',p')=\lbrace F'_1,F'_2,...,F'_m \rbrace $$
En matching funktion $Match$  tager disse to sæt af punkter med tilsvarende features, og opretter et sæt billede koordinater, som funktionen vurdere til at korrespondere. Matching funktionen kan beskrives som:
\begin{equation}
Match(\bold{P}, \bold{P}') = \bold{M}
\end{equation}
Hvor $\bold{P}, \bold{P}'$, er to sæt, hver bestående af $(F, p)$, og alle indgange i $\bold{M}$ består af par af koordinater, e.g. $(p_i, p'_j)$, vurderet til at korrespondere.
Hver feature kommer i form af en n-dimensional vektor, der beskriver interessepunktet udefra noget data. Denne vektor skal sammenlignes med vektorer i det modsvarende billede, for at finde det bedste match. Pga. ændringer i billederne vil korresponderende features aldrig være helt identiske, og der skal derfor oprettes nogle metoder, der kan determinere, hvornår to punkter er "ens" nok til at korrespondere.