\chapter{Abstract} \label{sec:abstract}
\vspace{2cm}
\begin{center}
\textbf{Resumé}
\end{center}
Korrespondanceanalyse af markbilleder er et led i en process, hvis formål er at relatere placering af ukrudt i enkelte markbilleder,
til et samlet ukrudtskort over marken. Korrespondanceanalysen består i at detektere interessepunkter i billedet, knytte deskriptorer til disse og derefter matche korresponderende interessepunkter. Fire detektorer (Difference og Gaussian, Determinant of Hessian, Harris og Morvec), to deskriptorer(SIFT og SURF) og en matching algoritme er beskrevet og implementeret. Metoden Determinant of Hessian og SURF, har givet den højeste repeatibility measure med en middelværdi på 0.1506, for 10 billeder, på samme højde. Metoden Harris og SIFT har givet det andenbedste resultat, med en middelværdi på 0.1505. Det er konkluderet at det i høj grad er muligt at opnå en konsistent korrespondanceanalyse af markbilleder, og at både hjørnedetektorer og blobdetektorer, kan bruges til at opnå korrespondancer.
\vspace{2cm}
\begin{center}
\textbf{Abstract}
\end{center}
Correspondanceanalysis of cropfield images, is part of a progress in which the purpose is to translate the locations of weed, in individual cropfield images, to a combined weed-map, of the entire field. The correspondance analysis consists of detecting interest points in each image, associate each of these with a descriptor and match corresponding interest points. Four detectors (Difference og Gaussian, Determinant of Hessian, Harris and Morvec), two descriptors (SIFT and SURF) and one matching algorithm, has been described and implemented. The method, Determinant of Hessian and SURF resulted in the highest repeatability measure, with an average of 0.1506 over ten cropfield images, taken from the same altitude. The method Harris and SIFT resulted in the second highest repeatability measure of 0.1505. The paper concludes that it is highly possible to consistently establish correspondances between cropfield images, and that cornerdetectors, as well as blobdetectors, can be used to establish correspondances between cropfield images.