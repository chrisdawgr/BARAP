\chapter{Konklusion}
Dette projekt blev tilgået, med ingen foregående viden/ide om hvorvidt det var muligt at opnå en korrekt korrespondanceanalyse. Der er nu afprøvet en række forskellige metoder, der alle bidrager til den samme konklusionen: At det i høj grad er muligt at opnå en korrekt korrespondanceanalyse, af markbilleder. Udover dette har undersøgelsen vist at ud af de afprøvede metoder, muliggør SURF den mest korrekte korrespondanceanalyse af markbilleder. 
Derudover konkluderes det, ud fra resultaterne, at metoderne skal være skalainvariante, men ikke rotationsinvariante, medmindre der forekommer flere rotationer imellem billederne, som i det isolerede tilfælde illustreret i figur \ref{fig:rota}.  \\ \\
Opgaven udleder hvilke metoder, der bedst kan anvendes til etableringen af ukrudtskortet, i projektet "Droner til monitering af flerårigt ukrudt i korn". Resultatet af denne opgave kan derfor bruges som en vejledning, for videreudvikling af projektet i praksis. 


% hvad kan man lære af opgaven hvad kan den bruges til (hvorfor er det vigtigt) 

% fejlkilder
% unanwsered question - sift

% kan resultaterne overføres til videre brug

