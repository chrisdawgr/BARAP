\section{Forbedringer}
Følgende afsnit vil beskrive hvilke forbedringer af den beskrevne implementering af korrespondance analyse af markbilleder, der kan foretages. 

\subsection{Non-maximal suppression}
Som beskrevet i SIFT afsnittet, og som gør sig gældende for både SIFT og SURF detektionsmetoderne, så er non-maximal suppression ikke implementeret for at finde sub-pixel nøjagtighed. 
\\
\\
I denne implementering bliver et punkt kun valgt, hvis $\hat{x} < 0.5$, i en given retning. $\hat{x}$ angiver en retningsvektor i tre dimensioner$(x, y, \sigma)$, der er orienteret mod et ekstrema og bruges i denne implementering til at frasortere punkter, der ikke er ekstremaer. Punkters placering er ikke blevet fundet ned til sub-pixel nøjagtighed - det blev ikke anset for en nødvendighed i denne udgave, da første prioritet var at finde korrespondancer og illustrere dem, mellem billeder - her er sub-pixel nøjagtighed ikke brugbart, da hver billedkoordinat er et naturligt heltal) - i en praktisk anvendelse, kan det ønskes, at der skal etableres homografier, eller bruge andre billedanalytiske metoder, på de fundne korrespondancer. Her er sub-pixel nøjagtighed strengt nødvendigt, for at få nøjagtige resultater. Implementering af non-maximal suppression som forklaret af Lowe, er derfor næste logiske udvidelse af programmet. 

\subsection{SIFT}
Givet grafen <??>, ses det, at den udregnede repeatability measure for SIFT er langt lavere, end nogle af de andre metoder. Det er lykkedes at finde korrespondancer med den samlede SIFT metode, der er dog langt færre matches, end i Harris/SIFT og SURF. Dette resultat er i uoverensstemmelse med andre sammenligninger af metoder\cite{kim} \cite{kim2}. Pedersen et al. har sammenlignet forskellige metoder under kontrollerede forhold, og med varierende grader mellem billederne, og beregnet en recall rate, der svarer til den repetability measure brugt her. Pedersen et al. opsætter nogle strengere krav(mere specifikt, tre krav) for, hvornår to punkter er potnetielt matchende ($C(I_1, I_2)$), end der gøres i dene opgave. Det er i studierne vist, at Harris og $DoG$ klarer sig bedre end en Hessian detektor. Dette resultat er i modstrid med resultaterne præsenteret her. Selvom applikationsområderne er forskellige, forventedes der alligevel en symmetri mellem resultaterne. Det må på baggrund af repeatability measure resultaterne konkluderes, at der er fejl i implementeringen af SIFT. 

\subsection{RGB til gråtone}
Som nævnt i afsnit 3.1, blev der anvendt en specifik metode, til at transformere intensiteter i RGB til gråtone værdier. Det kunne være interessant at undersøge, hvorvidt Excess Green(ExG) kan bruges til at øge repeatability measure. ExG kan bruges til at adskille planter fra jord\cite{exg}. Dette kunne skabe større kontrast i billedet og derfor mere tydelige blobs og hjørner. 