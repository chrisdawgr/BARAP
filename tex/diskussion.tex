\chapter{Diskussion}
I denne sektion vil resultater, fremført i afsnit \ref{sec:resultater}, fortolkes og kommenteres, for at besvare de opstillede problemstillinger. \\ \\
Først og fremmest var det interessant for dette projekt, at udlede hvorvidt det var muligt at opnå en korrekt korrespondanceanalyse af markbillederne, dvs. at udvælge korrekte korresponderende punkter. Opgavens hovedproblemstilling omhandler bestemmelsen af, hvilke af de udvalgte metoder, der bedst muliggør en korrekt korrespondanceanalyse af markbillederne. For at kunne konkludere ovenstående, iagttages to aspekter fra resultaterne: (1) Hvor mange korrekte korrespondancer, hver metode har fundet. (2) Hvor mange korrekte korrespondancer, hver metode har fundet, ift. mængden af de detekterede interessepunkter. Disse to aspekter er opstillet i nedenstående tabel og opsummere opgavens resultater.
\begin{center}
    \begin{tabular}{ | l | l | l | l |}
    \hline
    Detektor & Deskriptor & $\text{Match(A,A')}$ & $\text{mean rm}$ \\ \hline
    $DoH$ & SURF & TBD & 0.1209 \\ \hline  
    Harris & SIFT & TBD & 1.505 \\ \hline    
    Moravec & SIFT & 221.35 & 0.1025 \\ \hline    
    $DoG$ & SIFT & 1475.71 & 0.2342 \\ \hline         
    \end{tabular}
    \label{table:tab}
\end{center}
Tabellen viser at det var muligt for alle metoder at udvælge korrekte korrespondancer i billederne, og derved opnå en korrekt korrespondanceanalyse af markbillederne. For at besvare hovedproblemstillingen, evalueres metoderne ud fra deres repeatability measure $rm$. Det kan ud fra dette ses at metode kombinationen $DoG$ \& SURF, bedst muliggør en korrespondanceanalyse af markbillederne. Derudover kan det ses at den samlede metode SIFT, gav de dårligste resultater. Dette var overraskende, da SIFT er en metode der i høj grad er anerkendt og bredt anvendt. Denne opgaves implementering af SIFT er diskuteret i afsnit <forbedringer>. Resultaterne for hjørnedetektorene: Harris og Moravec, var overraskende gode og derfor kan det udledes at både hjørne - og blobdetektorer kan anvendes til korrespondanceanalyse af markbilleder.
\\ \\
For metoderne viste det sig at størstedelen af de brugbare punkter, i et stort omfang, blev udvalgt på én specifik oktav. Derfor viser resultaterne kun metoderne anvendt på den oktav, der gav de bedste resultater. Denne beslutning er yderligere beggrundet ved at dronen tager billederne fra én bestemt højde, der forekommer altså ikke højdeændringer i mellem de afprøvede billeder. Derfor kan det konkluderes at selvom størstedelen af de brugbare punkter blev udvalgt på samme oktav, var der behov for en skalarums repræsentation af markbillederne, for at kunne udvælge hvilken oktav, der gav de bedste resultater.
\\ \\
Som set i figur \ref{fig:rota} (øverst) er den ikke-rotationsinvariante deskriptor U-SURF, kun i stand til at etablere meget få korrekte korrespondancer, når billederne er roteret ift. hinanden. (Nederst) ses det at den rotationsinvariante deskriptor SURF er i stand til at udvælge alle korresponderende punkter korrekt. For de to billeder er det derfor nødvendigt at anvende en rotationsinvariant deskriptor, for at opnå en korrekt korrespondanceanalyse. De to udvalgte billeder repræsentere dog ikke resten af billedsættet, da der ikke forekommer en så markant rotation imellem de andre billeder. Derfor er det kun et isoleret tilfælde at deskriptoren skal være rotationsinvariant. Denne påstand er yderligere beggrundet ved at det også har været muligt at etablere et stort antal korrekte korrespondancer ved brug af U-
SURF.
\begin{figure}[H]
    \centering
    \includegraphics[width=1\textwidth]{fig/rot2.png}
     \vspace{-1em}
    \begin{center} 
       \caption{{\footnotesize \textit{Øverst: De 10 bedste matches fra $DoH$, U-SURF er illustreret. Nederst: De 10 bedste matches fra $DoH$ SURF er illustreret. }}}
    \label{fig:rota}
     \end{center}
     \vspace{-2.5em}
  \end{figure} \noindent
\section{Forbedringer}
Følgende afsnit vil beskrive hvilke forbedringer af den beskrevne implementering af korrespondance analyse af markbilleder, der kan foretages. 

\subsection{Non-maximal suppression}
Som beskrevet i SIFT afsnittet, og som gør sig gældende for både SIFT og SURF detektionsmetoderne, så er non-maximal suppression ikke implementeret for at finde sub-pixel nøjagtighed. 
\\
\\
I denne implementering bliver et punkt kun valgt, hvis $\hat{x} < 0.5$, i en given retning. $\hat{x}$ angiver en retningsvektor i tre dimensioner$(x, y, \sigma)$, der er orienteret mod et ekstrema og bruges i denne implementering til at frasortere punkter, der ikke er ekstremaer. Punkters placering er ikke blevet fundet ned til sub-pixel nøjagtighed - det blev ikke anset for en nødvendighed i denne udgave, da første prioritet var at finde korrespondancer og illustrere dem, mellem billeder - her er sub-pixel nøjagtighed ikke brugbart, da hver billedkoordinat er et naturligt heltal) - i en praktisk anvendelse, kan det ønskes, at der skal etableres homografier, eller bruge andre billedanalytiske metoder, på de fundne korrespondancer. Her er sub-pixel nøjagtighed strengt nødvendigt, for at få nøjagtige resultater. Implementering af non-maximal suppression som forklaret af Lowe, er derfor næste logiske udvidelse af programmet. 

\subsection{SIFT}
Givet grafen <??>, ses det, at den udregnede repeatability measure for SIFT er langt lavere, end nogle af de andre metoder. Det er lykkedes at finde korrespondancer med den samlede SIFT metode, der er dog langt færre matches, end i Harris/SIFT og SURF. Dette resultat er i uoverensstemmelse med andre sammenligninger af metoder\cite{kim} \cite{kim2}. Pedersen et al. har sammenlignet forskellige metoder under kontrollerede forhold, og med varierende grader mellem billederne, og beregnet en recall rate, der svarer til den repetability measure brugt her. Pedersen et al. opsætter nogle strengere krav(mere specifikt, tre krav) for, hvornår to punkter er potnetielt matchende ($C(I_1, I_2)$), end der gøres i dene opgave. Det er i studierne vist, at Harris og $DoG$ klarer sig bedre end en Hessian detektor. Dette resultat er i modstrid med resultaterne præsenteret her. Selvom applikationsområderne er forskellige, forventedes der en overførsel af 
\chapter{Konklusion}
Dette projekt blev tilgået, med ingen foregående viden/ide om hvorvidt det var muligt at opnå en korrekt korrespondanceanalyse. Der er nu afprøvet en række forskellige metoder, der alle bidrager til den samme konklusionen: At det i høj grad er muligt at opnå en korrekt korrespondanceanalyse, af markbilleder. Udover dette har undersøgelsen vist at ud af de afprøvede metoder, muliggør SURF den mest korrekte korrespondanceanalyse af markbilleder. Opgaven sætter fokus på hvilke metoder, der kan anvendes i etableringen af ukrudtskortet, til projekt "droner til monitering af ukrudt i marker". \\ \\ 
Derudover konkluderes det, ud fra resultaterne, at metoderne skal være skalainvariante, men ikke rotationsinvariante, medmindre der forekommer flere rotationer imellem billederne, som i det isolerede tilfælde illustreret i figur \ref{rotation}..

% hvad kan man lære af opgaven hvad kan den bruges til (hvorfor er det vigtigt) 

% fejlkilder
% unanwsered question - sift

% kan resultaterne overføres til videre brug

