\chapter{Eksperimental Opsætning}
Denne sektion har til formål at beskrive opgavens eksperimentale opsætning og dertil, hvilke billeder, der er anvendt og hvordan metoderne er afprøvet.
\section{Anvendte Billeder}
Det udvalgte billedsæt, som opgavens resultater bygger på, er taget fra overflyvningen af én mark. Sættene består af: 11 billeder som er taget længst væk fra jorden, 4 billeder taget næst-længst fra jorden og 4 billeder taget tættest fra jorden.
\begin{figure}[H]
    \centering
    \includegraphics[width=1\textwidth]{fig/43.png}
    \vspace{-0.5em}   
    \begin{center}
    \caption{{\footnotesize \textit{Udvalgt billedesæt.}}}
    \label{fig:lindblob}
     \end{center}
  \end{figure}
       \vspace{-2.7em}
\noindent
\section{Metode Opsætning}
Følgende er detektor/deskriptor kombinationer af beskrevne metoder, afprøvet på ovenstående billedsæt:
\begin{center}
    \begin{tabular}{ | l | l |}
    \hline
    Detektor & Deskriptor \\ \hline
    $DoG$ & SIFT  \\ \hline       
    Harris & SIFT \\ \hline    
    Moravec & SIFT \\ \hline    
    $DoH$ & SURF\\ \hline    
    \end{tabular}
\end{center}
Da der er blevet implementeret 4 detektorer og 2 deskriptorer, er der $4\times 2=8$ mulige metoder at afprøve. Det er her begrænset til de fire ovenstående kombinationer. SIFT er blevet valgt som deskriptor for metoderne, da vores undersøgelser har vist, at den giver flest matches og den højeste repeatability measure på de forskellige metoder, sammenlignet med SURF. 
\\ \\ 
Udover de ovenstående kombinationer af metoder, er nødvendigheden for rotationsinvarians undersøgt, ved at sammenligne resultater af en rotationsinvariant deskriptor: SURF, med en ikke-rotationsinvariaant deskriptor: U-SURF. Til det er der i billedsættet udvalgt par af billeder, hvor der opstår størst rotation imellem. Disse billeder er vist i figur \ref{fig:rotation}.
\begin{center}
    \begin{tabular}{ | l | l |}
    \hline
    Detektor & Deskriptor \\ \hline
    $DoG$ & U-SURF \\ \hline       
    $DoG$ & SURF \\ \hline     
    \end{tabular}
\end{center}
Ovenstående kombination af detektor/deskriptor afprøves kun imellem to billeder, hvor der foregår stor roation.
\section{Fejlkilder}
\subsection{Sortering af korrekte/ikke korrekte korrespondancer}
\section{Harris,Moravec Skala}
<FEJLKILDER>
<test kombination af billederne> \\
<hvad er testet, rotation, struktur, skala>