\section{SIFT}
SIFT\footnote{Distinctive Image Features
from Scale-Invariant Keypoints} er en korrespondanceanalyse metode, introduceret af David Lowe i 2004 \cite{SIFT}. SIFT er en samlet løsning, der består af en detektor og en deskriptor.
\section{Difference of Gaussian}
Difference of Gaussian, også referet til som "DoG", er en detektionsmetode introduceret  af David Lowe \cite{sift} og udgør feature detektoren i metoden SIFT\footnote{Distinctive Image Features
from Scale-Invariant Keypoints}. DoG er en skala-invariant blob detektor, der definere et billede i skalarummet det undersøgte billede foldet med et Gaussisk filter:
\begin{equation}
L(x,y,\sigma) \ast I(x,y)
\end{equation}
Metoden detektere punkter ved at lede efter ekstremaer i skala-rummet af DoG funktionen $ D(x,y,\sigma) $, hvilket kan udledes ved at tage forskellen imellem to nærliggende skalaer separeret af en en faktor $k$:

\begin{equation}
\begin{split}
D(x,y,\sigma) &= (G(x,y,k\sigma)-(G(x,y,\sigma))\ast I(x,y) \\
           &= L(x,y,k \sigma)-L(x,y,\sigma)
\end{split}
\end{equation}
DoG er en blob detektor, da det er en appproksimering af Laplacian of Gaussian eller "LoG" defineret som:
\begin{equation}
\begin{split}
LoG(x,y,\sigma) &= \sigma^2 \Delta^2L(x,y,\sigma) \\
                &= \sigma^2 (L_{xx}+L_{yy})
\end{split}
\end{equation}
Lowe relatere approksimeringen af DoG og LoG, ved varme-diffusion ligningen:
\begin{equation}
\dfrac{\partial L}{\partial \sigma} = \sigma \Delta^2L
\label{heat}
\end{equation}
Diffusions ligningen fra \eqref{heat}, kan approksimeres ved:
\begin{equation}
\begin{split}
\dfrac{\partial L}{\partial \sigma} &= \sigma \Delta^2L \\
&= \lim_{k \to 0} \dfrac{L(x,y,k\sigma)-L(x,y,\sigma)}{k\sigma-\sigma} \\
&\approx \dfrac{L(x,y,k\sigma)-L(x,y,\sigma)}{k\sigma-\sigma}
\end{split}
\label{difference}
\end{equation}
Yderligere kan \eqref{difference} udledes til,
\begin{equation}
\begin{split}
(k\sigma-\sigma)\sigma\Delta L &\approx L(x,y,k,\sigma)-L(x,y,\sigma) \\
(k-1)\sigma^2\Delta L &\approx L(x,y,k\sigma)-L(x,y,\sigma) \\
(k-1)\sigma^2LoG &\approx DoG
\end{split}
\end{equation}


\subsection{Orientering af SIFT punkter}
Formålet ved dette skridt er at tildele hvert punkt en orientering.
\\
\\
Et $16\times 16$ dataindsamlingsvindue placeres på interessepunktets skalabillede, omkring interessepunktet. For alle punkter i dataindsamlingsvinduet, er størrelsen af punkternes gradient, og deres orientering beregnet ved:
\begin{equation}
m(x,y) = \sqrt{(L(x + 1, y) - L(x - 1, y))^2 + (L(x, y + 1) - L(x, y - 1))^2} 
\label{magnitudepoint}
\end{equation}
\begin{equation}o(x,y) = tan^{-1}((L(x,y+1) - L(x,y-1))/(L(x+1, y) - L(x-1, y))) 
\label{orientationpoint}
\end{equation}
hvor $m$ er størrelsen af en gradient, og $o$ er gradientens retning. For alle punkter indenfor dataindsamlingsvinduet, udregnes ligning
\eqref{magnitudepoint}, \eqref{orientationpoint}, og danner et gradientvindue $g$ og orienteringsvindue $v$, begge vinduer af størrelse $16\times16$. Gradientvinduet skal herefter foldes med et Gaussisk filter hvor $\sigma_{Gauss} = 1.5 \cdot \sigma_{point}$, med størrelse $16\times16$. Dette udføres for at vægte gradienter nær punktet højere end punkter i udkanten af vinduet.  $\sigma_{Gauss}$ er sigmaværdien tilhørende Gaussfiltret, der foldes med gradientvinduet. $\sigma_{point}$ er sigmaværdien i $DoG$ billedet, punktet er fundet på. 
\\
\\
Der oprettes derefter et orienteringshistogram $H$, med 36 indgange. Orienteringerne angivet i $v$ skal tilføjes histogrammet, vægtet af gradientstyrken angivet i $g$. En indgang i $H$, dækker en vinkel på $10^{\circ}$. F.eks. skal alle gradienter med vinkler mellem  $0^{\circ}-10^{\circ}$, tilføjes vægtet, til $H_1$, osv. Dette resulterer i et orienteringshistogram, hvor indgangen i histogrammet, med størst værdi, bliver bearbejdet. Lowe foreslår, at alle indgange i histogrammet, der ligger indenfor 80\% af det højeste punkt, bliver nye features - dette er undladt her, for at reducere antallet af features. 
\\
Der skal nu foretages en interpolation, omkring den indgang i histogrammet, med størst værdi, for at få et mere præcist estimat af $\theta$. Toppunktet af andengradspolynomiet vil udgøre orienteringen der tildeles punktet. 
\begin{figure}[H]
    \centering
    \includegraphics[width=0.60\textwidth]{fig/sift-orientation-histogram.jpg}
     \vspace{-1em}
    \begin{center}    
       \caption{\textcolor{gray}{\footnotesize \textit{Histogrammet $H$ afbilledet, sammen med en andengradsligning. Andengradsligningen er beregnet over den største indgang i $H$}}}
    \label{histogramheight}
     \end{center}
     \vspace{-2.5em}
  \end{figure} \noindent
Dette ses på figur \ref{histogramheight}, hvor et andengradspolynomium er blevet tilnærmet den største værdi af $H$. Estimering af andengradspolynomiet sker over den største værdi i $H$, og dens venstre og højre nabo.


\subsubsection*{Algoritme: SIFT orientering af interessepunkter}
\begin{enumerate}
\item[Input:] Billede $I$
\item[] Interessepunkter $p  \in (x, y)$
\item[Output:]  Interessepunkter tildelt orientering $p \in (x,y, \theta)$
\end{enumerate}

\begin{enumerate}
\item Et dataindsamlingsvindue placeres omkring interessepunktet, på det skalabillede interessepunktet er fundet på.
\item Ligning \eqref{magnitudepoint}, \eqref{orientationpoint} anvendes på hvert punkt i dataindsamlingsvinduet, for at udregne gradient retninger og størrelser. Gradientbilledet foldes derefter med et Gaussisk filter, hvor $\sigma_{Gauss} = 1.5 \cdot \sigma_{point}$
\item Værdierne i $g$ adderes på indgange i histogrammet $H$, afhængigt af deres orientering i $v$.
\item  Der udføres en interpolation omkring den største værdi i $H$ og to naboer. Denne værdi returneres.
\end{enumerate}

\subsection{Deskriptor}
SIFT deskriptoren skaber invarians overfor skala, belysning og vinkel. Der benyttes et datainsamlingsvindue $W$, der har størrelsen $16x16$, til at indsamle information omkring et interessepunkt, og denne information udtrykkes som en vektor, med 128 indgange. Deskriptoren kan formelt beskrives:
\begin{equation}
Des(I, p,\sigma,\theta) = F
\end{equation}
Hvor $I$ er billedet, $p$ er et interessepunkt $(x,y)$, $\sigma$ er den tilhørende skalaparameter punktet er fundet på, og $\theta$ er orienteringen af $p$. $F$ er en feature vektor, tilegnet punktet $p$.
\\
\\
Skalaen for et givent punkt bestemmer, om det billede der skal foretages beregninger på, skal være subsamplet. For at opnå rotationsinvarians, bliver dataindsamlingsvinduet $W$ roteret i forhold til $\theta$. Dette opnås, ved at tage prik produktet, af hver indgang i dataindsamlingsvinduet, med rotationsmatricen:
\begin{equation}
W_{{mn}_{new}} = W_{mn} \cdot
\begin{pmatrix}
  \cos \theta & -\sin \theta \\
  \sin \theta & \cos \theta  \\
\end{pmatrix}
\label{rotaionmatrix}
\end{equation}
$W_{{mn}_{new}}$ bruges til at indsamle information om størrelsen og orienteringen af gradienten i det $i$'ende punkt, som i \eqref{magnitudepoint} og \eqref{orientationpoint} respektivt - gradienten og orienteringen bliver gemt i to separate matricer af størrelse $16\times16$, hhv $Grad$ og $Ori$. $Grad$ bliver herefter foldet med et Gaussisk filter størrelse $16\times16$, hvor sigmaværdien er halv så stor, som størrelsen af vinduet ($=8$):
\begin{equation}
Grad_{new} = G(x,y,8) * Grad(x,y)
\label{gradientsmooth}
\end{equation}
Herefter skal $G_{new}$ inddeles i 16 regioner ($R_{ij} \subseteq (G_{new}$), hver med størrelse $4\times4$, som set i figur(???). Der skal foretages trilineær interpolation, og alle de kontinuerte orienteringer i $O$, skal vægtes som beskrevet nedenfor, og værdierne i figur (???, b), der svarer til $F_i$, skal opdateres - her er $F_i$ repræsenteret ved 16 vinduer, hver med 8 retninger (svarede til 45 grader). Dette sker i 3 skridt:
\begin{enumerate}
\item{ alle punkter i $R_{ij}$ skal vægtes efter, hvor langt de ligger fra centrum af $R_{ij}$. Det er vedtaget, at der eksisterer tre længder (beskriv v. billede!!!)}
\item{ 4 indgange i $F_i$ opdateres efter punkterne i $R_{ij}$'s orientering. Punkterne i $R_{ij}$ skal fordeles på de 4 af de 8 orienteringer, der ligger tættest orienteringen af $R_{ij}$. Disse skal vægtes med $G_{new}$ og værdien fundet i 1. }
\item{ De 4 indgange, der er blevet opdateret, skal tilsvarende opdateres i 3 naboområder, af $F_i$ (hvis der eksisterer nogen)}
\end{enumerate}

$F$ normaliseres, og alle værdier større end 0.2 skal sættes til 0.2, og $F$ skal normaliseres igen.
\subsection*{Algoritme}
\begin{enumerate}
\item Det bestemmes via $\sigma$ hvilket billede der skal udføres beregninger på
\item $W$ roteres, ved rotationsmatricen \eqref{rotaionmatrix}, og glattes med \eqref{gradientsmooth}
\item $R_{ij}$ oprettes, ved at dele $Grad_{new}$ op i 16 regioner, hver med størrelse $4\times4$
\item Alle punkter, i alle regioner af $R_{ij}$, skal nu bruges til at opdatere $F$, som beskrevet de 3 skridt ovenfor
\end{enumerate}