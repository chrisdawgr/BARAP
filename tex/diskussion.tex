\chapter{Diskussion}
I denne sektion vil resultater, fremført i afsnit \ref{sec:resultater}, fortolkes og kommenteres, for at besvare de opstillede problemstillinger. \\ \\
Først og fremmest var det interessant for dette projekt, at udlede hvorvidt det var muligt at opnå en korrekt korrespondanceanalyse af markbillederne, dvs. at udvælge korrekte korresponderende punkter. Opgavens hovedproblemstilling omhandler bestemmelsen af, hvilke af de udvalgte metoder, der bedst muliggør en korrekt korrespondanceanalyse af markbillederne. For at kunne konkludere ovenstående, iagttages to aspekter fra resultaterne: (1) Hvor mange korrekte korrespondancer, hver metode har fundet. (2) Hvor mange korrekte korrespondancer, hver metode har fundet, ift. mængden af de detekterede interessepunkter. Disse to aspekter er opstillet i nedenstående tabel og opsummere opgavens resultater.
\begin{center}
    \begin{tabular}{ | l | l | l | l |}
    \hline
    Detektor & Deskriptor & $\text{Match(A,A')}$ & $\text{mean rm}$ \\ \hline
    $DoH$ & SURF & TBD & 0.0292 \\ \hline  
    Harris & SIFT & TBD & 1.505 \\ \hline    
    Moravec & SIFT & TBD & TBD \\ \hline    
    $DoG$ & SIFT & TBD & 0.2342 \\ \hline         
    \end{tabular}
    \label{table:tab}
\end{center}
Tabellen viser at det var muligt for alle metoder at udvælge korrekte korrespondancer i billederne, og derved opnå en korrekt korrespondanceanalyse af markbillederne. For at besvare hovedproblemstillingen, evalueres metoderne ud fra deres repeatability measure $rm$. Det kan ud fra dette ses at metode kombinationen $DoG$ \& SURF, bedst muliggør en korrespondanceanalyse af markbillederne. Derudover kan det ses at den samlede metode SIFT, gav de dårligste resultater. Dette var overraskende, da SIFT er en metode der i høj grad er anerkendt og bredt anvendt. Denne opgaves implementering af SIFT er diskuteret i afsnit <forbedringer>. Resultaterne for hjørnedetektorene: Harris og Moravec, var overraskende gode og derfor kan det udledes at både hjørne - og blobdetektorer kan anvendes ved korrespondanceanalyse af markbilleder.
\\ \\
For metoderne viste det sig at størstedelen af de brugbare punkter, i et stort omfang, blev udvalgt på én specifik oktav. Derfor viser resultaterne kun metoderne anvendt på den oktav, der gav de bedste resultater. Denne beslutning er yderligere beggrundet ved at dronen tager billederne fra én bestemt højde, der forekommer altså ikke højdeændringer i mellem de afprøvede billeder. Derfor kan det konkluderes at selvom størstedelen af de brugbare punkter blev udvalgt på samme oktav, var der behov for en skalarums repræsentation af markbillederne, for at kunne udvælge hvilken oktav, der gav de bedste resultater.
\\ \\
Som set i figur <>, er den ikke-rotationsinvariante deskriptor U-SURF, ikke i stand til at etablere korrekte korrespondancer, når billederne er roteret ift. hinanden. Dette lykkedes dog for den rotationsinvariante deskriptor SURF. For de to billeder er det derfor nødvendigt at anvende en rotationsinvariant deskriptor, for at opnå en korrekt korrespondanceanalyse. De to udvalgte billeder repræsentere dog ikke resten af billedsættet, da der ikke forekommer en så markant rotation imellem de andre billeder. Derfor er det kun et isoleret tilfælde at deskriptoren skal være rotationsinvariant. Denne påstand er yderligere beggrundet ved at det også har været muligt at etablere et stort antal korrekte korrespondancer ved brug af U-SURF.
\\ \\
Resultaterne viser også at både hjørner og blobs succesfuldt kan anvendes i udvælgelsen af interessepunkter.
%Hjørner