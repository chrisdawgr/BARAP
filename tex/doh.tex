\subsection{Determinant of Hessian}
Determinant of Hessian eller \textit{DoH} udgør feature detektoren i
SURF. DoH er baseret på Hessian matricen, som udregnes for hvert punkt $p$ i et billede:
\begin{equation}
\mathcal{H}(p, \sigma) = 
 \begin{bmatrix}
 	L_{xx}(p, \sigma) & L_{xy}(p, \sigma) \\
 	L_{xy}(p, \sigma) & L_{yy}(p, \sigma) 
 \end{bmatrix}
 \label{hessianmatrix}
\end{equation}
hvor $\sigma$ angiver skala og $L_{xy} $, $L_{yy}$ og $L_{xx}$ (vist i ligning\eqref{lxx}), er den Gaussiske funktionen, partielt differentieret ift. $xy$, $yy$ og $xx$.
\begin{equation}
L_{xx}(x, \sigma) = (\frac{\partial^2 }{\partial x^2 } G(x,y,\sigma)) * I
\label{lxx}
\end{equation}
Bay et al. anvender approkismerede boks filtre: $D_{xx}$, $D_{yy}$ og $D_{xy}$, der udelukkende består af værdierne $-2,-1,0, 1$ som fordelagtigt, ved brug af integralbilleder, nedsætter antallet af beregninger drastisk. I denne implementering er disse boks filtre ikke anvendt, og er derved udregnet som ligning \eqref{lxx}. Figur \ref{fig:lxxlyylxy} er en illustration af de afledte Gaussiske filtre, der er anvendt.
\begin{figure}[H]
    \centering
    \includegraphics[width=0.75\textwidth]{fig/31.png}
     \vspace{-0.5em}
    \begin{center}    
       \caption{\textcolor{gray}{\footnotesize \textit{ }}}
    \label{fig:lxxlyylxy}
     \end{center}
     \vspace{-2.5em}
  \end{figure} \noindent
I SURF oprettes et skalarum, der ligesom i SIFT opdeles i skalaer og oktaver. I stedet for at reducere billedernes størrelser, forøges størrelserne af de anvendte filtre. SIFT anvender en stigende sigma værdi, med en fast filterstørrelse. Dette er det samme som at anvende en stigende filterstørrelse med en fast sigma værdi. I figur \ref{fig:siftsurf} ses forholdet imellem størrelserne af den Gaussiske funktion.
\begin{figure}[H]
    \centering
    \includegraphics[width=0.75\textwidth]{fig/34.png}
     \vspace{-0.5em}
    \begin{center}    
       \caption{\textcolor{gray}{\footnotesize \textit{Øverst ses den gaussiske funktion, hvor der ligesom i SIFT er vedligeholdt en filterstørrelse, men med stigende sigma værdi. Nederst ses den samme funktion, der ligesom i SURF vedligeholder den samme sigma værdi, men anvender stigende filterstørrelser.}}}
    \label{fig:siftsurf}
     \end{center}
     \vspace{-1.5em}
  \end{figure} \noindent
For hver oktav anvendes der, i denne implementering, fire billeder foldet med fire forskellige filterstørrelser. For hver oktav, bruges nr. 2 filterstørrelse, fra forrige oktav til starten af næste, og størrelsen imellem filtre, fordobles fra forrige oktav, som vist i tabel \ref{fig:secderivfiltersize}. Her skal det bemærkes, at de andenafledte filtre, skal have dobbelt størrelse, af de førsteafledte.
\begin{figure}[H]
    \centering
    \begin{center}    
    \begin{tabular}{ | l | l | l | l | l |}
    \hline
    oktav & filter str 1 & filter str 2 & filter str 3 & filter str 4 \\ \hline
    1 & 9 & 15 & 21 & 27 \\ \hline
  	2 & 15 & 27 & 39 & 51 \\ \hline
  	3 & 27 & 51 & 75 & 99 \\ \hline
  	4 & 51 & 99 & 147 & 195 \\ \hline
    \end{tabular}       
    \caption{\textcolor{gray}{\footnotesize \textit{Fire forskellige oktaver, og filterstørrelse, for et andenafledt filter}}}
    \label{fig:secderivfiltersize}
     \end{center}
     \vspace{-2.5em}
  \end{figure} \noindent
For hvert punkt i billederne opstilles Hessian matricen, hvorefter determinanten udregnes for alle punkter i billederne. Bay et.al. udregner determinanten ved:
\begin{equation}
\textbf{det}\mathcal{H}_{approksimeret} = D_{xx}D_{yy}-wD_{xy}^2
\label{deerminantofhessian}
\end{equation}
hvor $w$ er en vægt, tilføjet for at balancere de approksimerede differentierede Gaussiske filtre og boks filtrene. Da denne implementering ikke anvender boks filtre, udregnes determinanten som:
\begin{equation}
\textbf{det}\mathcal{H} = L_{xx}L_{yy}-L_{xy}^2
\label{deerminantofhessian}
\end{equation}
Hessian matricen anvendes, da den beskriver den lokale geometriske krumning omkring et punkt. Hessian matricens egenvektorer angiver de principielle retninger af krumningerne, og egenværdierne beskriver størrelserne af disse krumninger. For blobs ønskes et lokalt ekstrema, og derved at egenværdierne enten begge er positive eller begge er negative. Er dette ikke gældende vil blobben have en krumning i hver sin retning og derved være et saddle-point. Determinanten udregnes, da det er produktet af egenværdierne og derfor fortæller noget om fortegnet på disse. En positiv determinant vil betyde at egenværdierne er af samme fortegn, hvor en negativ vil betyde at fortegnet for egenværdierne er forskellige og punktet derfor placeret på et saddle-point. I figur \ref{fig:konkav}(venstre) ses et punkt placeret på et maxima, hvor krumningen er ens i alle retningerne derfor vil egenværdierne være negative og determinanten positiv. I det andet billede (højre) ses et punkt placeret på et saddle-point. Krumningerne er forskellige, hvilket vil resultere i egenværdier af forskellige fortegn og derved en negativ determinant. 
\begin{figure}[H]
    \centering
    \includegraphics[width=0.55\textwidth]{fig/36.png}
     \vspace{-0.5em}
    \begin{center}    
       \caption{\textcolor{gray}{\footnotesize \textit{Til venstre ses et punkt placeret på et maksima. Til højre ses et punkt placeret på et saddle-point}}}
    \label{fig:konkav}
     \end{center}
     \vspace{-1.5em}
  \end{figure} \noindent
Metoden reagere derved på mørke og lyse blobs, ved positive determinanter. Derfor udvælges et lokalt maxima af et $3\times3\times3$ område af determinant billederne, som vist i figur \ref{fig:difference}. Dette udføres for alle oktaver. Herefter udvælges korrekt subpixel placering, hvilket er en metode introduceret af David Lowe, også beskrevet i SIFT. Igen udføres dette skridt ved at fjerne punkter, der ikke er lokaliseret tæt nok på ekstremaer.
\subsection*{Algoritme}
\begin{enumerate}
\item {Hessian matricen opstilles for alle punkter i skalabillederne som i ligning \eqref{hessianmatrix}}
\item Determinantbilleder opstilles, ved at udregne determinanten af Hessian matricen for alle punkter i alle billeder som i ligning \eqref{deerminantofhessian}.
\item Lokale maxima udvælges, af hvert $3\times3\times3$ område af billeder på samme oktav.
\item Accurate keypoint localisation, bruges til at fjerne dårligt lokaliserede punkter ligesom i ligning \eqref{maxsurp}.
\end{enumerate}