\section{Detektor}\label{sec:detect}
Feature detektion er en metode indenfor billedbehandling, der for hvert punkt $p = (x,y)$ i et billede $I$, bestemmer om punktet er et interessepunkt. Denne kan beskrives ved:
\begin{equation}
Detect(I)= \bold{P}
\label{detect}
\end{equation}
hvor $\bold{P} = \lbrace p_1, p_2,..., p_n \rbrace$ er et sæt punkter i billedet $I$, udvalgt til at være interessepunkter. Moravec \cite{moravec} definerer, som en af de første, et interessepunkt(feature) således: "A
feature is good if it can be located unambiguously in different views of a scene. A
uniformly colored region or a simple edge does not make for good features because
its parts are indistinguishable". Iflg. Moravec skal et interessepunkt have egenskaber, der gør det utvetydigt lokaliserbart. Denne definition, sammen med Lindebergs \cite{pointsurvey} er blevet adopteret, og et interessepunkt karakteriseres her ved:
\begin{itemize}
\item{\emph{Repeterbar}: Repeterbarheden significere uafhængighed af forskellige betingelser i billedet. Dvs. at samme interessepunkt skal kunne findes i to forskellige billeder, trods ændringer af f.eks lys og rotation.}
\item{\emph{Informationsrigt:}
Intensiteten i og omkring interessepunktet skal være unikt, så punktet kan skelnes fra de punkter, den ikke korresponderer med.}
\item{\textit{Definerbar struktur:} Punktet skal indgå i en struktur, der er matematisk definerbar. Dette er nødvendigt da punkterne skal kunne identificeres ved en algoritme.}
\end{itemize}
En Detektor skal være robust overfor små geometriske og fotometriske deformationer samt støj. Geometriske forskelle kan omfatte rotation, skalering og ændring af perspektiv.  Fotometriske forskelle kan omfatte ændringer i belysning (f.eks. ved delvist overskyet vejr) eller forskelle i refleksion mod de to drone-positioner, hvorfra billederne er taget. Der er bred enighed om nødvendigheden for skalainvarians hos deteketorer \cite{koen} \cite{blob} \cite{lindenscale}. Følgende afsnit beskriver, hvordan en sådan invarians kan opnås.
\raggedbottom\subsection{Skalarum}
En Detektor skal være robust overfor små geometriske og fotometriske deformationer samt støj. Geometriske forskelle kan omfatte rotation, skalering og ændring af perspektiv.  Fotometriske forskelle kan omfatte ændringer i belysning (f.eks. ved delvist overskyet vejr) eller forskelle i refleksion mod de to drone-positioner hvorfra billederne er optaget. En god detektor skal være invariant over for sådanne forskelle. Der er bred enighed om nødvendigheden for skalainvarians hos deteketorer \cite{koen} \cite{blob} \cite{lindenscale}. Dette afsnit beskriver, hvordan en sådan invarians kan opnås. \\ \\
Objekter i virkeligheden, såvel som  af-billedet, optræder kun som meningsfulde enheder over et specifikt skalainterval. Et træ vil indenfor centimeters eller nanometers afstand optræde som blade eller molekyler og indenfor meters afstand som et træ. 
%Koenderink \cite{koen} definere et objekts skalainterval som værende afgrænset af to skalaer, hvor objektet er def: den inderste og den yderste skala, f.eks. vil en trætop have en indre skala på ca. 10 cm og en ydre skala på ca. 10 m. %Da billedet indeholder en ukendt scene, 
Det vides ikke hvilket skalainterval, der er meningsfuldt, for et givent objekt i en scene. 
%Den inderste skala af billedet vil derfor være bundet af pixelstørrelsen og den ydre af billedets fysiske størrelse.
En aksiomatisk tilgang til dette problem er at undersøge et bredt skalainterval af billedet, i form af en udvidelse af billedfunktionen fra ligning \eqref{bf} med en enkelt parameter også kaldt skalaparametren $\sigma$:
\begin{equation}
\begin{split}
&L: R^3 \rightarrow R \\
&L(x,y,\sigma) = \lambda \hspace{0.5 cm} (x,y,\sigma)\in R^3, \lambda \in [1,256] \subset \mathbb{R}
\end{split}
\end{equation}
% Bekræft søren
For at opnå en multi-skala repræsentation af billedet, oprettes der et skalarum bestående af en stak skalabilleder, der går fra at udtrykke finere til grovere strukturer, proportionelt med skalaparametren, som illustreret i figur \ref{fig:scalerep}. 
\begin{figure}[H]
    \centering
    \includegraphics[width=0.55\textwidth]{fig/32.png}
     \vspace{-1em}
    \begin{center}    
       \caption{\textcolor{gray}{\footnotesize \textit{ }}}
    \label{fig:scalerep}
     \end{center}
     \vspace{-2.5em}
  \end{figure} \noindent
Denne overgang fra finere til grovere strukturer kan opnås ved systematisk at folde billederne med et Gaussisk filter af stigende $\sigma$ værdi, hvor det nederste billede i stakken kan repræsenteres ved biledet $ L(x,y,0) = I(x,y)$ og for $\sigma>0$, defineres et skalabillede $I_\sigma(x,y)$ ved:
\begin{equation}
L(x,y,\sigma) = G(x,y,\sigma)\ast I(x,y)
\label{scalespace1}
\end{equation}
\\
Filteret der bruges til at oprette en skalarumsrepræsentation, skal opfylde følgende egenskaber:
\begin{itemize}
\item{Et lokalt ekstrema må ikke forøges når skala parametren stiger. Dvs. at intensiteten hos et maksima eller minima ikke må respektivt forøges eller formindskes. Dette medfører at nye detaljer ikke må tilføjes, når skalaparametren stiger \cite{lindkth}.}
\item{Hvis hver udglatnings kerne er tilknyttet en skalaparameter, og to kerner foldes med hinanden, ønskes det at den resulterende kerne har en skalaparameter, lig med summen af de foregående skalaparametre \cite{springer}.
\begin{equation}
g(\cdot;\sigma_1) \ast g(\cdot;\sigma_2)=g(\cdot;\sigma_1+\sigma_2)
\label{semi}
\end{equation}
Ovenstående definere at alle skalarums transformationer er af samme familie. En skalarums repræsentation ved en grov skala $\sigma_2$, kan derved udledes ved en repræsentation fra en finere skala:
\begin{equation}
g(\cdot;\sigma_2) = g(\cdot;\sigma_2-\sigma_1)\ast g(;\sigma_1)\text{     hvor     }\sigma_2>\sigma_1
\end{equation}}
\end{itemize}
Der anvendes et Gaussisk filter da den imødekommer begge ovenstående egenskaber. Ved et Gaussisk filter vil strukturer, der eksistere på grovere skalaer være simplificeringer af strukturer fra finere skalaer. Der kan derved ikke opstå nye strukturer ud af ingenting ved anvendelse af det Gaussiske filter. Witkin \cite{witkins} beviste dette i tilfældet for én-dimensionsalle signaler illustreret i figur \ref{scalespace1}, der viser resultatet af at glatte et signal med et Gaussisk filter af stigende $\sigma$ værdi. Det ses tydeligt, hvordan signalets underliggende grovere struktur bliver udtrykt og at finere strukturer undertrykkes.
\begin{figure}[H]
    \centering
    \includegraphics[width=0.45\textwidth]{fig/33.png}
     \vspace{-1em}
    \begin{center}    
       \caption{\textcolor{gray}{\footnotesize \textit{Et én-dimensionalt signal udsat for et Gaussisk filter af gradvist stigende størrelse.}}}
    \label{fig:scalereps}
     \end{center}
     \vspace{-2.5em}
  \end{figure} \noindent
<Måske billede fra lindenscale du ved hvad jeg snakker om Chris>
\subsubsection*{Skala Pyramide}
En udbredt metode, hvorpå skalarummet kan repræsenteres, er ved en \textit{skalapyramide}. Ved en skalapyramide repræsentation oprettes en pyramide af kopier af det undersøgte billede, hvor der for hvert niveau i pyramiden sker en reduktion af billedets størrelse, med en faktor af to. 
Når billedets størrelse skal halveres, er det ikke nok bare at fjerne, hver anden række og hver anden kolonne, da dette vil resultere i store tab af billedets informationer. For at imødekomme dette problem, foldes billedet med et Gaussisk filter, inden billedet reduceres. <Et gaussisk filter anvendes da det flader billedet ud, hvilket formindsker tabet af information>. <Generelt ønskes det at sigma værdien for et Gaussiske filter er fordoblet for hver reduktion af billedet.>
 \begin{figure}[H]
    \centering
    \includegraphics[width=0.40\textwidth]{fig/40.png}
     \vspace{-1em}
    \begin{center}    
       \caption{\textcolor{gray}{\footnotesize \textit{ }}}
    \label{fig:scalerepdiff}
     \end{center}
     \vspace{-2.5em}
  \end{figure} \noindent
Hvis $G_0=I(x,y)$, kan de forskellige niveauer $l$ i skalapyramiden opnås ved:
\begin{equation}
G_l(i,j)=\sum\limits_{m}\sum\limits_{n}w(m,n)G_{l-1}(2i+m,2j+n)
\end{equation}
hvor $w$ er et Gaussisk filter. Fordelene ved en pyramiderepræsentation er at billedernes størrelser reduceres, hvilket reducere antallet af beregninger drastisk.
\raggedbottom\subsection{Strukturer}
Interessepunkter defineres ikke udefra semantisk meningsfulde objekter, som ansigter eller bøger, da dette vil kræve en høj-niveau fortolkning af scenen. I stedet anvendes lav-niveau strukturere, der identificeres i lokale pixelområder, der er matematisk definerbare. Nedenstående er en gennemgang af forskellige lokale lav-niveau strukturere, der kan bruges i udvælgelsen af interessepunkter.
\subsection{Hjørner}\label{subsec:corner}
En hjørnedetektor er en feature detektor, der definere et hjørne som værende et interessepunkt. Et hjørne kan defineres ved et punkt, der har to dominerende kanter i hver sin retning, derved er et hjørne et punkt, hvor der foregår store intensitetskift i det omrkingliggende område. Som følge af denne definition, er et hjørne distinktivt og kan bruges som interessepunkt. At detektere hjørner er en udbredt teknik indenfor feature detektion, da hjørner ofte forekommer i forskellige menneskeskabte scener og fordelagtigt kan bruges i disse sammenhæng.
\begin{figure}[H]
    \centering
    \includegraphics[width=0.55\textwidth]{fig/6.png}
    \vspace{-1em}   
    \begin{center}    
    \caption{\textcolor{gray}{\footnotesize \textit{
     Tre udvalgte vinduer, med interessepunkter i centrum af samme motiv. \textbf{(a)} Punktet er lokaliseret i en teksturløs region, d.v.s. ingen teksturskift. \textbf{(b)} Punktet er lokaliseret på en kant. \textbf{(c)} Punktet er lokaliseret på et hjørne }}}
    \label{fig:2}
     \end{center}
    \vspace{-2.7em}  
  \end{figure}  
\noindent
En intuitiv måde at definere hvorfor et hjørne er interessant, er at placere et rektangulært vindue omkring punktet. Dette vindue forskydes lokalt i x og y retningen. Opstår der et nyt objekt, identisk med interessepunktet, som resultat af forskydningen er punktet ikke distinktivt og derfor ej interessant. I figur \ref{fig:2} ses tre udvalgte punkter med et rektangulært vindue placeret over. I stil med ovenstående definition, forskydes det firkantede vindue i alle retninger. Forskydes \textbf{(a)} vil det ligne alle de forskudte billeder da punktet og regionen omkring er homogent. Punktet er derfor ikke interessant. Forskydes \textbf{(b)} i x-aksen opnås et nyt objekt, men en forskydning i y-aksen vil resultere i samme objekt af en kant, og punktet er derfor ikke interessant. Punktet placeret på et hjørne \textbf{(c)} er interessant da ingen forskydninger vil resultere i det originale billedet. Hjørnet kan derfor bruges som et interessepunkt. Denne intuitive definition kan kvantificeres til en matematisk definition, der estimere auto-korrelationen imellem de forskudte billeder, hvilket angiver intensitetsskiftene imellem billederne og derved, hvor der opstår et hjørne. 
\subsection{Blobs}
En blob er en region i et billede, hvor intensitet er konstant, og forskellig fra intensiteten udenfor regionen. Lindenberg \cite{blob} definere blobs som værende lyse regioner på sort baggrund eller omvendt - altså strukturer, der står i kontrast til deres baggrund. 

%Det lokale ekstrema gør blobben til en veldefineret, lav-niveau struktur og kravet om konstant intensitet gør den pr. definition distinktiv. 
\begin{figure}[H]
    \centering
    \includegraphics[width=0.35\textwidth]{fig/11.png}
    \vspace{-0.5em}   
    \begin{center}
    \caption{\textcolor{gray}{\footnotesize \textit{
    En blob visualiseret i 2-d, udefra Lindenberg's definition \cite{blob}}}}
    \label{fig:lindblob}
     \end{center}
  \end{figure}
       \vspace{-2.7em}
\noindent
%MEGA FORKERT
%MEGA FORKERT
%MEGA FORKERT
%MEGA FORKERT
I figur \ref{fig:lindblob}(a), ses en blob defineret af dens lokale ekstrema, hvor styrken af blobben beksrives ved kontrasten, ift. området omkring ekstremaet. Lindenberg definere en blob som værende afgrænset af dens saddelpunkt; et saddelpunkt angiver punktet, hvor intensiteten stopper med at falde og starter med at stige for lyse blobs, og modsat for mørke.
%MEGA FORKERT
%MEGA FORKERT
%MEGA FORKERT
%MEGA FORKERT
%MEGA FORKERT
%MEGA FORKERT
%MEGA FORKERT
\\
\\
En metode til at detektere ekstremaer, er Laplace af Gauss(LoG). Laplace defineres således:
\begin{equation}
\Delta f = \nabla^2 f =  \sum_{i = 1}^n \frac{\partial^2 f}{\partial x^2_i}
\end{equation}
Anvendes Laplace operatoren på Gaussfunktionen (ligning \eqref{2dgaussian}), kan resultatet diskretiseres og anvendes som en kerne.
\begin{equation}
LoG= \sigma^2\Delta G
\label{lap}
\end{equation}
Bemærk, at $\sigma^2$ er blevet multipliceret på fra venstre. Dette er gjort for at normalisere LoG, således, at responset er invariant overfor størrelsen af $\sigma$. LoG kan diskritiseres og foldes med billedet for at finde lokale ekstremaer. \\

\begin{figure}[H]
    \centering
    \includegraphics[width=0.90\textwidth]{fig/normLoG.jpg}
    \vspace{-0.5em}   
    \begin{center}
    \caption{\textcolor{gray}{\footnotesize \textit{
    (a) En 3-D visualisering af en to-dimensional Laplacian of Guassian (b) ét en-dimensionalt signal (c) Laplacian of Gaussian operatoren anvendt på (b)}}}
    \label{fig:normLoG}
     \end{center}
  \end{figure}
       \vspace{-2.5em}
\noindent
Når blobs skal lokaliseres, skal $\sigma$ værdien være tilpasses blobbens størrelse, som ses på figur \eqref{fig:normLoG}. Figuren viser fire forskellige værdier for $\sigma$. I første illustration er $\sigma$ værdien for lav - her dannes flere ekstremaer, men ingen af dem karakteriserer en blob, da den absolutte værdi for funktionen evalueret på det afledte af LoG signalet, er for lav. Dog ser kurven i illustration tre ud til, at have tilstrækkelig høj absolut værdi, til at kunne karakteriseres som et blob. \\
Problemet omkring valg af skala, kan afhjælpes ved skalarumsrepræsentation, som gennemgås i.

\begin{figure}[H]
    \centering
    \includegraphics[width=0.25\textwidth]{fig/29.png}
    \vspace{-0.5em}   
    \begin{center}
    \caption{\textcolor{gray}{\footnotesize \textit{
    }}}
    \label{fig:scale}
     \end{center}
  \end{figure}
       \vspace{-2.5em}
\noindent
I figur \ref{fig:scale} angiver cirklerne forskellige undersøgte skalaer, Så hvordan udvælges cirklen, der dækker interesse området uafhængigt af områdets størrelse?  For Blobs er det interessant når der i et skaleret område opstår et veldefineret ekstrema. En måde at søge efter ekstremaer over forskellige skalaer er ved at oprette et skalarum for det undersøgte billede, hvor hvert billede skaleres og der for hver skala findes interessepunkter. Skala-rummet i et 2-dimensionalt billede repræsenteres af flere billeder i forskellige skalaer af det originale billede. Billeder, der repræsentere forskellige skalaer, opnås ved at folde billedet iterativt med et Gaussisk filter med stigende $\sigma$ værdi. 
\begin{figure}[H]
    \centering
    \includegraphics[width=0.65\textwidth]{fig/24.png}
    \vspace{-0.5em}   
    \begin{center}
    \caption{\textcolor{gray}{\footnotesize \textit{
Til venstre ses en visualisering af et skala-rum formet som en pyramide. Hvert niveau angiver en skala repræsentation af det originale vindue, hvor toppen af pyramiden indeholder billeder af største skala og derfor med mindst information, og bunden af skalaen med det originale billede. Til højre ses et billede foldet med et Gaussisk filter af stigende sigma værdier. Jo højere sigma værdi, jo flere fine detaljer bliver fjerne og billedet slørret.
    }}}
    \label{fig:mona}
     \end{center}
  \end{figure}
       \vspace{-2.5em}
\noindent
Et Gaussisk filter bruges da gradvis højere værdier af $\sigma$ fjerner fine strukturer, som vist i figur \ref{fig:mona}, og nye strukturer forekommer ikke ved transformationen fra finere til grovere skalaer \cite{lindenscale}. Idéen er derved at fjerne disse strukturer og lede efter  andre ekstremaer, gradvist på større skalaer, der også kan detekteres.
Et billede i skalarummet for billedet $f(x,y)$ kan derfor defineres som i \eqref{scalespace}
\begin{equation}
L(x,y,\sigma) = G(x,y,\sigma)\ast f(x,y)
\label{scalespace}
\end{equation}
hvor $G$ er det 2-dimensionelle Gaussiske filter,$L(x,y,\sigma)$ repræsentere et et billede i skala-rummet, og skala-parametren $\sigma$, bestemmer skalaen, eller placeringen i skala-rummet. $L(x,y,0) = f(x,y)$, da det er den "nederste" skala og den nederste del af pyramiden. Højere niveauer af pyramiden kan opnås ved at folde billedet med et Gaussisk filter af større sigma værdi.