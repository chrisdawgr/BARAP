\section{Matching}
Når featuresne er fundet, skal der bruges matching til at estimere, om to punkter korrespondere. Bemærk, at $\boldsymbol{\sigma}$(fed) i dette afsnit, betegner standardafvigelsen, hvor $\sigma$ er skala, som hidtil. Matching kan beskrives som:
\begin{equation}
Match(\bold{P}, \bold{P}') = \bold{M}
\end{equation}
Hvor $\bold{P}, \bold{P}'$, er to lister, bestående af $(F, p, \sigma)$ (altså feature, punkter($x,y$) og skala), og alle indgange i $M$ består af par af koordinater, eg. $(p_i, p'_j)$.
\\
Der eksisterer mange måder, der kan matches på. Måden der foretages matching på her, foregår i to trin: Først bregnes den euklidiske afstand, og derefter, sorteres dårlige matches fra:

\begin{equation}
||P|| = \sqrt{\sum\limits_{i=1}^n(F_i-F'_i)^2}
\label{euc}
\end{equation}

For hver eneste feature i $\bold{P}$, beregnes alle afstandene, til features i $\bold{P}'$, og den feature i $\bold{P}'$ der ligger nærmest, bliver valgt og herved dannes et par, der tilføjes til det midlertidige sæt, af matches. Når alle inputs er blevet matches, sorteres de efter afstand, med korteste afstand først. 
\\
Ved brug af ovenstående, bliver der altså ikke sorteret dårlige matches fra, og hvis $|\bold{P}| > |\bold{P}'|$, er det garanteret, at mindst 1 element(og muligvis flere) fra $\bold{P}'$, har flere matches, fra $\bold{P}$. Dette er naturligvis ikke hensigtsmæssigt, da det antyder, at minimum 1 af matchesne, er matched forkert. Der er altså brug for en metode, til at sortere dårlige matches fra.
\\
\\
En tilgang kunne være, udelukkende at kigge på de $m$ bedste matches. Empiriske undersøgelser har vist, at denne tilgang virker med $m=20$. Her fremkommer ingen outliers. Jo større $m$ bliver, des flere outliers kommer der. Der skal altså bruges en metode, der kan sortere outliers fra - specielt kunne det være interessant at vide, hvor mange inliers, der bliver fundet i alt.
\\
For at løse dette, kan en statistiske udregninger sammen med motion-vektorene bruges. 
\\
I denne metode, fjernes matches, der har en euklidiske afstand, der er mindre end middelværdien, af de 10 bedste matches, multipliceret med 5. Det resulterende sæt af punkters motion-vektor, skal nu udregnes: $MV = p_i - p'_i$. 
$MV$'s standard afvigelse beregnes nu i x og y retningen. For x kan dette skrives: 
\begin{equation}
\boldsymbol{\sigma_x} = \sqrt{ \sum \limits_{n=1}^N (x_i  - \mu)^2 }
\end{equation}
og beregningen for y analog med (31). \\
En iterativ beregning behøves nu, for at fjerne punkter, hvis afstand fra middelværdispunktet, er større, end standardafvigelsesafstanden, her opskrevet som en indikatorfunktion:



\[
indikator=
\begin{cases}
    1,& \sqrt{(x_i - \mu_x)^2 + (y_i - \mu_y)^2} < \sqrt{\boldsymbol\sigma_x^2 + \boldsymbol\sigma_y^2} \\
    0,& \text{ellers}
    \end{cases}
\]
\\
\\
Ovenstående bliver gentaget, indtil en thresholdværdi for standardafvigelsen er nået. Processen er illustreret, på billede(????). Her ses, hvordan hver iteration fjerer støjpunkter, og bevæger sig tættere på en klynge af punkter, der må formodes at være inliers.