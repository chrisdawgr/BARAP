\section{Matching}
Fra deskriptoren er et sæt interessepunkter $\bold{P}$ beskrevet ved et sæt af features $\bold{F}$ 
En matching funktion $Match$ tager disse to sæt $ (\bold{P},\bold{F})$, hver bestående af 
$ (p_i,f_i) $
Funktionen tager disse to sæt af punkter med tilsvarende features, og returnere et sæt parvise punkter vurderet til at korrespondere. Matching funktionen kan beskrives som:
\begin{equation}
Match(\mathbb{P}, \mathbb{P}') = \bold{M}
\end{equation}
Hvor alle indgange i $\bold{M}$ består af parvise koordinater, e.g. $(p_i, p'_j)$. Feature vektorene i det ene billede skal sammenlignes med feature vektorer i det andet billede. Pga. ændringer i billederne vil korresponderende features aldrig være helt identiske, og der skal derfor oprettes nogle metoder, der kan determinere, hvornår to Features er ens nok til at tilsvarende punkter korrespondere.