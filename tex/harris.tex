\section{Harris}\label{sec:harris}
Harris hjørnedetektor er introduceret af Chris Harris og Mike Stephens \cite{harris}. Detektoren kan ses som en videreudvikling af Moravec, da den adresserer Moravecs begrænsninger:
\begin{enumerate}
\item{ \textit{Anisotropisk respons:} Harris og Stephens udvider Moravecs udregning af auto-korrelation til at måle intensitetsvariationer i alle retninger. Auto-korrelationen imellem skiftende vinduer kan approksimeres ved en Taylor udvidelse af anden orden:
\begin{subequations}
\begin{align}
E(u,v) = & \sum w(x,y)[I(x + u, y + v) - I(x,y)]^2 \\
\approx & \sum w(x,y)[I(x,y) + uI_x  + vI_y - I(x,y)]^2 \\
= & \sum w(x,y)[u^2I_x^2 + 2uvI_xI_y + v^2I_y^2]  \label{last}
\end{align}
\end{subequations}
Ligning \eqref{last} kan omskrives til matrixformen
\begin{equation}
E(u,v) \approx
\begin{bmatrix}
        u & v
     \end{bmatrix}
M
\begin{bmatrix}
        u \\
        v
     \end{bmatrix}
\end{equation} 
Hvor $M$ er auto-korrelations matricen (også kaldt strukturtensoren), hvor $w(x,y)$  er et Gaussisk filter:
\begin{equation}
M = w(x,y) 
\begin{bmatrix}
	I_x^2 & I_xI_y \\
	I_xI_y & I_y^2
\end{bmatrix}
\label{structens}
\end{equation}
Billedets afledte i $x$ og $y$ retning findes ved at udføre en differentiering af billedet, hvilket kan opnås
 ved at folde billedet, med henholdsvis $[-1 \hspace{0.1cm} 0 \hspace{0.1cm} 1]$ og $[-1 \hspace{0.1cm} 0 \hspace{0.1cm} 1]^T$. }
\item{\textit{Støj sensitiv:} Harris detektoren anvender et Gaussisk vindue $w(x,y)$ omkring det undersøgte pixelområde. Det Gaussiske vindue anvendes til at fjerne støj fra billedet.}
\item{\textit{Højt kant respons} Strukturtensoren 'M' opskrevet i \eqref{structens} indeholder de kvadrerede differentierede retninger i et givet punkt. Egenværdierne for denne matrix, vil være proportionale med de forskellige steder vinduet kan placeres:
\begin{itemize}
\item{ \textit{Homogen flade.} Billedintensiteten er konstant. Begge egenværdier vil være små.}
\item{\textit{Kant.} For punkter placeret på en kant, vil én egenværdi være stor, mens den anden vil være lille}
\item{\textit{Hjørne.} Et punkt placeret på et hjørne, vil have to store egenværdier.}
\end{itemize}
Det ønskes derfor, for at identificere hjørner, at begge egenværdier for M er store. Egenværdierne kan defineres ud fra determinanten og sporet af M:
\begin{subequations}
\begin{align}
\textbf{det}M & = \lambda_1 \lambda_2 \\
\textbf{trace}M & = \lambda_1+\lambda_2
\end{align}
\end{subequations}
Harris og Stephens foreslår et mål for hjørnestyrken, baseret på forholdet imellem de to egenværdier. En høj værdi for hjørnestyrken angiver to store egenværdier. Forholdet mellem egenværdierne kan defineres ved:
\begin{equation}
R = \textbf{det}M-k(\textbf{trace}M)^2
\label{rvalharris}
\end{equation}
hvor \textit{k} er en empirisk defineret konstant udledt af Harris og Stephens som ligger i intervallet $[0.04,0.06]$. I figur \ref{fig:egen} ses sammenhængen mellem værdien $R$, egenværdierne og hvordan de relatere sig til hjørner, kanter og homogene områder, igen kan det ses at for et hjørne ønskes to store egenværdier.
\begin{figure}[H]
    \centering
    \includegraphics[width=0.45\textwidth]{fig/26.png}
     \vspace{-1em}
    \begin{center}    
       \caption{{\footnotesize \textit{Sammenhængen imellem strukturtensorens egenværdier.}}}
    \label{fig:egen}
     \end{center}
     \vspace{-2.5em}
  \end{figure} \noindent
Ved værdien for R, kan det defineres om et punkt er lokaliseret på et hjørne. De interessante punkter, er hvor der opstår en stor hjørnestyrke, hvilket kan findes ved at sætte en grænseværdi $t$, der definere om et hjørne er lokaliseret(sandt/falsk).
\begin{equation}
\begin{split}
\texttt{hjørne} = 
\begin{cases}
\texttt{sandt}& \text{hvis } R\geq t, \\
\texttt{falsk }& \text{hvis } R < t.
\end{cases}
\end{split}
\label{cornerhar}
\end{equation}
}
\end{enumerate}
\subsubsection*{Algoritme: Harris Detektor}
\begin{tabbing}
Input\quad \= : \= Billede $I$\\
Output \text{ } \> : \> Interessepunkter $p \in (x,y)$
\end{tabbing}
\begin{enumerate}
\item{For hver pixel i billedet $(x,y)$, udregnes strukturtensoren, beskrevet i ligning \eqref{structens}.}
\item{For hvert pixel i billedet udregnes hjørnestyrken ved ligning \eqref{rvalharris}.  Hvis den udregnede hjørnestyrke opfylder ligning \ref{cornerhar}, for en given grænseværdi, udvælges punktet som værende et interessepunkt.}
\end{enumerate}
\subsection*{Konklusion}
Metoden bruges i høj grad i dag. Detektoren er rotationsinvariant pga. dens brug af egenværdier, men er ifølge Mikolajczyk og Schmid \cite{eval}, stadig delvis anisotropisk, hvilket skyldes differentierings filtrene, der kun udledes i $x$ og $y$ retningen. Mikolajczyk og Schmid \cite{eval1} konkludere at detektoren er særlig følsom overfor rotation på omkring 45 grader. De konkludere også at detektorens repeterbarhed falder drastisk ved skalaændringer.