\section{Detektor}\label{sec:detect}
Feature detektion er en metode indenfor billedbehandling, der for hvert punkt $p = (x,y)$ i et billede $I$ bestemmer, om dette punkt er et interessepunkt, hvis egenskaber kan karakteriseres som\cite{pointsurvey}::
\begin{itemize}
\item{\emph{Repeterbar}: Repeterbarheden significere uafhængighed af forskellige betingelser i billedet. Dvs. at samme interessepunkt skal kunne findes i to forskellige billeder, trods ændringer af f.eks lys og rotation}
\item{\emph{Distinktive:}
De fundne interessepunker skal være unikke ift. intensitetsvariationen i det omkringliggende område.} 
\end{itemize}
En feature detektor '$Detec$' er en funktion, der får $I$ som input, og returnere interesspunkter.
$$ Detec	(I)= \bold{P}$$
Moravec \cite{moravec} som en af de første, definerede et interessepunkt således: "A
feature is good if it can be located unambiguously in different views of a scene. A
uniformly colored region or a simple edge does not make for good features because
its parts are indistinguishable. Regions, such as corners, with high contrast in
orthogonal directions are best." I følge Moravec, indeholder et godt interessepunkt altså stor kontrast, i de ortogonale retninger - den stringente matematiske definition, vil blive gennemgået i senere afsnit. \\
\\
Moravec er blot en definition af et interessepunkt ud af mange, og i sidste ende angiver anvendelsesområdet, hvilke punkter, der lokaliseres bedst. Uanset hvilke strukturer der anvende.
\\
\\
Imellem to billeder, kan der være geometriske forskelligheder, som f.eks. rotation, skalering(se figur  \ref{fig:skal}) ændring af perspektiv. Der kan også forekomme fotometriske ændringer, som ændring af pixelintensitet(ændring i R,G og/eller B værdien i en pixel) forsaget af lysændringer. Disse faktorer, skal en detektor helst være invariant overfor.
\begin{figure}[H]
    \centering
    \includegraphics[width=0.37\textwidth]{fig/28.png}
     \vspace{-1em}
    \begin{center}    
       \caption{\textcolor{gray}{\footnotesize \textit{To ens kurver, set på forskellige to To ens kurver, set på forskellige skalaer. Opfattes kurven langt væk, som den store kurve, vil de udvalgte områder opfattes som kanter. Opfattes hele kurven vil området vise et hjørne. Det kan derfor, alt efter applikationsdomænet, være nødvendigt at anvende en skala invariant detektor.}}}
    \label{fig:skal}
     \end{center}
     \vspace{-2.5em}
  \end{figure} \noindent
Detektoren skal også være robust overfor små deformationer som støj i billedet, hvilket fejlagtigt kan fortolkes som interessepunkter. Interessepunkter defineres ikke udfra semantisk meningsfulde områder, som ansigter eller bøger, da dette vil kræve en høj-niveau fortolkning af scenen. I stedet anvendes lav-niveau strukturere, der identificeres i lokale pixelområder, og er matematisk definerbare. Nedenstående er en gennemgang af forskellige lokale lav-niveau strukturere, der kan bruges i udvælgelsen af interessepunkter.
\subsection{Hjørner}\label{subsec:corner}
En hjørnedetektor er en feature detektor, der definere et hjørne som værende et interessepunkt. Et hjørne kan defineres ved et punkt, der har to dominerende kanter i hver sin retning, derved er et hjørne et punkt, hvor der foregår store intensitetskift i det omrkingliggende område. Som følge af denne definition, er et hjørne distinktivt og kan bruges som interessepunkt. At detektere hjørner er en udbredt teknik indenfor feature detektion, da hjørner ofte forekommer i forskellige menneskeskabte scener og fordelagtigt kan bruges i disse sammenhæng.
\begin{figure}[H]
    \centering
    \includegraphics[width=0.55\textwidth]{fig/6.png}
    \vspace{-1em}   
    \begin{center}    
    \caption{\textcolor{gray}{\footnotesize \textit{
     Tre udvalgte vinduer, med interessepunkter i centrum af samme motiv. \textbf{(a)} Punktet er lokaliseret i en teksturløs region, d.v.s. ingen teksturskift. \textbf{(b)} Punktet er lokaliseret på en kant. \textbf{(c)} Punktet er lokaliseret på et hjørne }}}
    \label{fig:2}
     \end{center}
    \vspace{-2.7em}  
  \end{figure}  
\noindent
En intuitiv måde at definere hvorfor et hjørne er interessant, er at placere et rektangulært vindue omkring punktet. Dette vindue forskydes lokalt i x og y retningen. Opstår der et nyt objekt, identisk med interessepunktet, som resultat af forskydningen er punktet ikke distinktivt og derfor ej interessant. I figur \ref{fig:2} ses tre udvalgte punkter med et rektangulært vindue placeret over. I stil med ovenstående definition, forskydes det firkantede vindue i alle retninger. Forskydes \textbf{(a)} vil det ligne alle de forskudte billeder da punktet og regionen omkring er homogent. Punktet er derfor ikke interessant. Forskydes \textbf{(b)} i x-aksen opnås et nyt objekt, men en forskydning i y-aksen vil resultere i samme objekt af en kant, og punktet er derfor ikke interessant. Punktet placeret på et hjørne \textbf{(c)} er interessant da ingen forskydninger vil resultere i det originale billedet. Hjørnet kan derfor bruges som et interessepunkt. Denne intuitive definition kan kvantificeres til en matematisk definition, der estimere auto-korrelationen imellem de forskudte billeder, hvilket angiver intensitetsskiftene imellem billederne og derved, hvor der opstår et hjørne. 
\section{metoder indenfor billedbehandling}\label{subsec:kant}
Formelt kan et billede repræsenteres som en funktion af to variable:
\begin{equation}
\begin{split}
&I: \mathbb{\mathbb{Z^+}}^2 \rightarrow \mathbb{Z^+} \\
&I(x,y) = \lambda \hspace{0.5 cm} (x,y)\in \mathbb{Z^+}^2, \lambda \in [1,256] \subset \mathbb{Z^+}
\end{split}
\end{equation}
hvor $\lambda$ repræsentere en billedintensitet, også kaldet pixelværdi. $\lambda$ er defineret, indenfor grænserne af billedet og er 0 udenfor, dvs.: 
\begin{equation}
 I(x, y) =
\begin{cases}
    \lambda, & \text{hvis } 1 \leq x \leq x_{max}, 1 \leq y \leq y_{max} \\
    0,              & \text{ellers}
    \label{pixelintensitet}
\end{cases}
\end{equation}
En kort bemærkning bør gøres om det anvendte billedformatet. De undersøgte billeder er af filtype JPEG(Joint Photographic Experts Group) og hver pixelintensitet indeholder originalt 8x3 bits information til hhv. rød, grøn og blå: $\lambda_{col} = [R,G,B]^T \in \mathbb{R}^3$. Hver farve kan antage $2^8 = 256$ værdier og hver værdi ligger i intervallet $[0,1]$. Disse værdier bliver transformeret til en gråtone værdi, vha. Lumosity metoden <cite>:
\begin{equation}
Lum(\lambda_{col}) = [0.2126, 0.7152, 0.0722] \lambda_{col} = \lambda
\label{lumosity}
\end{equation}
Hver gang pixelværdi eller billedintensitet benævnes, er det underforstået, at de har undergået den lineære transformation, ligning \cite{lumosity}. $x$ og/eller $y$ kan højst antage værdien 65.535, og et billede er derfor i sagens natur diskret.
\\
\\
Diskontinuitet i billeder er ofte en nyttig egenskab, i en billedanalytisk procedure. Det kan være i form af en kant: Hvis et billede beskues i 3D, kan en kant illustreres i 1D, ved et snit af et billede vinkelret på overfladen, illustreret figur \ref{fig:kant}. En kant er en lokal ændring i det afledte signal $f$.
\noindent
\begin{figure}[H]
    \centering
    \includegraphics[width=0.55\textwidth]{fig/7.png}
     \vspace{-1em}
    \begin{center}        
     \caption{\textcolor{gray}{\footnotesize \textit{
En 1-dimensional fortolkning af intensiteten i et billede. Intensitetsskiftet i midten (ved x $approx$ 1000, er en kant.}}}
    \label{fig:kant}
     \end{center}
       \vspace{-2.5em}
  \end{figure}
\noindent
En differentiering af funktionen fra figur \ref{fig:kant} vil fremhæve dens udsving og derved angive hvor der forekommer kanter. Differentiering af billeder kan approsikmeres ved følgende ligning:
\begin{equation}
\dfrac{df(x)}{dx}=\dfrac{f(x+1)-f(x-1)}{2}
\label{diff}
\end{equation}
\\
\\
<beskriv hvorfor foldning er nyttigt>
Foldning af $I$ af størrelse $(M \times N)$, med en kerne $K$ der har størrelse $(m \times n)$, hvor $M > k, N > n$:
\begin{equation}
O(i,j) = \sum\limits_{x=1}^m \sum\limits_{y=1}^n I(i+k-1,l-1)K(k,l)
\label{foldning}
\end{equation}
Ligning \eqref{foldning} udregnes for alle $i,j \in I$. En kerne defineres her som en matrix af arbitrær dimension - ofte $(N\times N)$. 
\\
\\
Et billede kan differentieres, som anført ligning \eqref{diff}, ved brug af foldning. Først defineres en kerne til differentiering i $x$-aksen $K$, som: $K = [\frac{1}{2}, 0, \frac{1}{2}]$. Kernen foldes med billedet $I$: $I \ast K $, hvor $\ast$ udgør foldningsoperatoren.
\\
\\
Når en kerne skal foldes med et billede hvor afstanden til kanten af billedet, er skarpt mindre, end størrelsen af kernen, gælder ligning \eqref{pixelintensitet} for $I$.
\\
\\
Det kan være problematisk at lokalisere kanter vha. differentiering. I figur \ref{fig:kant}, vil støj i billedet(de små udsving) også blive fremhævede. For at fjerne støjen, kan billedet foldes med et Gaussisk filter, hvilket er en diskret approksimering til den Gaussiske funktion. Foldning af et billede med et Gaussisk filter vil resultere i en flydende overgang af pixelværdierne og derfor glatte billedet. <sæt gaussbillede ind> Den Gaussiske funktion i 2-D, hvor $ \sigma $ er standardafvigelsen af den Gaussiske fordelingen, er defineret som:
\begin{equation}
G(x,y,\sigma) = \frac{1}{2 \pi \sigma ^{2}} e^{- \frac{x^{2} + y^{2}}{2 \sigma ^{2}}}
\label{2dgaussian}
\end{equation} 
For at undgå først at glatte billedet ved at folde med et Gaussisk filter, og derefter folde med et differentieringsfilter udnyttes det, at foldning er en associativ operation:
\begin{equation}
\dfrac{\partial}{\partial x}(G \ast f) = (\dfrac{\partial}{\partial x}G) \ast f
\end{equation}
Her er $G$ er det Gaussiske filter, men kunne være en vilkårlig anden kerne, og $f$ et signal. 
\\
Foldes et differentieret 1-dimensionelt Gaussfilter med signalet fra figur \ref{fig:kant}, vil det resultere i et bakkeformet signal, hvor bakken indikere en kant. For en mere lokaliserbar kant, kan den dobbelt afledte tages, som set i figur \ref{fig:deriv}. I sidstnævnte tilfælde, kan kanten lokaliseres, hvor funktionen krydser nul.
\begin{figure}[H]
    \centering
    \includegraphics[width=0.55\textwidth]{fig/8.png}
    \vspace{-1em}   
    \begin{center}
    \caption{\textcolor{gray}{\footnotesize \textit{
     Resultatet af at folde et dobbelt differentieret Gaussisk filter med funktionen}}}
    \label{fig:deriv}
     \end{center}
    \vspace{-2.5em}  
  \end{figure}
\noindent
I de metoder der i denne opgave er gjort brug af, er diskontinuiteter i lokale strukturer i billederne undersøgt. Derfor er det nyttigt at bruge et dataindsamlingsvindue $\bold{D}$, da dette kan repræsentere et lokalområde. Et dataindsamlingsvindue er en $(N\times N)$ matrix over en funktion af et udsnit af et billede. For et dataindsamlingsvindue over koordinaterne $(x,y)$ tilhørende $I$ gælder, at $(x,y)$ er centrum i dataindsamlingsvinduet, dvs:
$$
\bold{D_{\frac{N}{2},\frac{N}{2}}} = f(I(x,y))
$$
de andre værdier i $\bold{D}$, er resultatet af en funktion over $I$, omkring $(x,y)$ - dette kan være identitetsfunktionen, men også være en udregning af gradienter.
\subsection{Blobs}
En blob er en region i et billede, hvor intensitet er konstant, og forskellig fra intensiteten udenfor regionen. Lindenberg \cite{blob} definere blobs som værende lyse regioner på sort baggrund eller omvendt - altså strukturer, der står i kontrast til deres baggrund. 

%Det lokale ekstrema gør blobben til en veldefineret, lav-niveau struktur og kravet om konstant intensitet gør den pr. definition distinktiv. 
\begin{figure}[H]
    \centering
    \includegraphics[width=0.35\textwidth]{fig/11.png}
    \vspace{-0.5em}   
    \begin{center}
    \caption{\textcolor{gray}{\footnotesize \textit{
    En blob visualiseret i 2-d, udefra Lindenberg's definition \cite{blob}}}}
    \label{fig:lindblob}
     \end{center}
  \end{figure}
       \vspace{-2.7em}
\noindent
%MEGA FORKERT
%MEGA FORKERT
%MEGA FORKERT
%MEGA FORKERT
I figur \ref{fig:lindblob}(a), ses en blob defineret af dens lokale ekstrema, hvor styrken af blobben beksrives ved kontrasten, ift. området omkring ekstremaet. Lindenberg definere en blob som værende afgrænset af dens saddelpunkt; et saddelpunkt angiver punktet, hvor intensiteten stopper med at falde og starter med at stige for lyse blobs, og modsat for mørke.
%MEGA FORKERT
%MEGA FORKERT
%MEGA FORKERT
%MEGA FORKERT
%MEGA FORKERT
%MEGA FORKERT
%MEGA FORKERT
\\
\\
En metode til at detektere ekstremaer, er Laplace af Gauss(LoG). Laplace defineres således:
\begin{equation}
\Delta f = \nabla^2 f =  \sum_{i = 1}^n \frac{\partial^2 f}{\partial x^2_i}
\end{equation}
Anvendes Laplace operatoren på Gaussfunktionen (ligning \eqref{2dgaussian}), kan resultatet diskretiseres og anvendes som en kerne.
\begin{equation}
LoG= \sigma^2\Delta G
\label{lap}
\end{equation}
Bemærk, at $\sigma^2$ er blevet multipliceret på fra venstre. Dette er gjort for at normalisere LoG, således, at responset er invariant overfor størrelsen af $\sigma$. LoG kan diskritiseres og foldes med billedet for at finde lokale ekstremaer. \\

\begin{figure}[H]
    \centering
    \includegraphics[width=0.90\textwidth]{fig/normLoG.jpg}
    \vspace{-0.5em}   
    \begin{center}
    \caption{\textcolor{gray}{\footnotesize \textit{
    (a) En 3-D visualisering af en to-dimensional Laplacian of Guassian (b) ét en-dimensionalt signal (c) Laplacian of Gaussian operatoren anvendt på (b)}}}
    \label{fig:normLoG}
     \end{center}
  \end{figure}
       \vspace{-2.5em}
\noindent
Når blobs skal lokaliseres, skal $\sigma$ værdien være tilpasses blobbens størrelse, som ses på figur \eqref{fig:normLoG}. Figuren viser fire forskellige værdier for $\sigma$. I første illustration er $\sigma$ værdien for lav - her dannes flere ekstremaer, men ingen af dem karakteriserer en blob, da den absolutte værdi for funktionen evalueret på det afledte af LoG signalet, er for lav. Dog ser kurven i illustration tre ud til, at have tilstrækkelig høj absolut værdi, til at kunne karakteriseres som et blob. \\
Problemet omkring valg af skala, kan afhjælpes ved skalarumsrepræsentation, som gennemgås i.

\begin{figure}[H]
    \centering
    \includegraphics[width=0.25\textwidth]{fig/29.png}
    \vspace{-0.5em}   
    \begin{center}
    \caption{\textcolor{gray}{\footnotesize \textit{
    }}}
    \label{fig:scale}
     \end{center}
  \end{figure}
       \vspace{-2.5em}
\noindent
I figur \ref{fig:scale} angiver cirklerne forskellige undersøgte skalaer, Så hvordan udvælges cirklen, der dækker interesse området uafhængigt af områdets størrelse?  For Blobs er det interessant når der i et skaleret område opstår et veldefineret ekstrema. En måde at søge efter ekstremaer over forskellige skalaer er ved at oprette et skalarum for det undersøgte billede, hvor hvert billede skaleres og der for hver skala findes interessepunkter. Skala-rummet i et 2-dimensionalt billede repræsenteres af flere billeder i forskellige skalaer af det originale billede. Billeder, der repræsentere forskellige skalaer, opnås ved at folde billedet iterativt med et Gaussisk filter med stigende $\sigma$ værdi. 
\begin{figure}[H]
    \centering
    \includegraphics[width=0.65\textwidth]{fig/24.png}
    \vspace{-0.5em}   
    \begin{center}
    \caption{\textcolor{gray}{\footnotesize \textit{
Til venstre ses en visualisering af et skala-rum formet som en pyramide. Hvert niveau angiver en skala repræsentation af det originale vindue, hvor toppen af pyramiden indeholder billeder af største skala og derfor med mindst information, og bunden af skalaen med det originale billede. Til højre ses et billede foldet med et Gaussisk filter af stigende sigma værdier. Jo højere sigma værdi, jo flere fine detaljer bliver fjerne og billedet slørret.
    }}}
    \label{fig:mona}
     \end{center}
  \end{figure}
       \vspace{-2.5em}
\noindent
Et Gaussisk filter bruges da gradvis højere værdier af $\sigma$ fjerner fine strukturer, som vist i figur \ref{fig:mona}, og nye strukturer forekommer ikke ved transformationen fra finere til grovere skalaer \cite{lindenscale}. Idéen er derved at fjerne disse strukturer og lede efter  andre ekstremaer, gradvist på større skalaer, der også kan detekteres.
Et billede i skalarummet for billedet $f(x,y)$ kan derfor defineres som i \eqref{scalespace}
\begin{equation}
L(x,y,\sigma) = G(x,y,\sigma)\ast f(x,y)
\label{scalespace}
\end{equation}
hvor $G$ er det 2-dimensionelle Gaussiske filter,$L(x,y,\sigma)$ repræsentere et et billede i skala-rummet, og skala-parametren $\sigma$, bestemmer skalaen, eller placeringen i skala-rummet. $L(x,y,0) = f(x,y)$, da det er den "nederste" skala og den nederste del af pyramiden. Højere niveauer af pyramiden kan opnås ved at folde billedet med et Gaussisk filter af større sigma værdi.