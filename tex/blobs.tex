\subsection{Blobs}
En blob er en sammenhængende region af pixels, der enten er betydeligt lysere eller mørkere end dens omgivelser, altså strukturer der står i kontrast til deres baggrund. En blob kan derfor udvælges som værende et ekstrema i billedet. blobbens omfang strækker sig til punktet hvor den møder en anden blob. Dette kan intuitivt illustreres for lyse-blobs på en mørk baggrund, ved at betragte billedets funktion som et landskab dækket af vand. Sænkes vandet vil toppe vises. Sænkes vandet yderligere vil to toppe forbindes, 'dalen' hvor de to toppe mødes vil definere deres omfang. Et sådant punkt hedder et saddle-point, som set i figur \ref{fig:lindblob} og definere et punkt der har krumninger i hver sin retning.
\begin{figure}[H]
    \centering
    \includegraphics[width=0.35\textwidth]{fig/11.png}
    \vspace{-0.5em}   
    \begin{center}
    \caption{\textcolor{gray}{\footnotesize \textit{
    En blob visualiseret i 2-d, udefra Lindenberg's definition \cite{blob}}}}
    \label{fig:lindblob}
     \end{center}
  \end{figure}
       \vspace{-2.7em}
\noindent
En blob kan udvælges ved at identificere den geometriske krumning i om omkring et punkt ved de principielle krumninger. De to principielle krumninger i et punkt angiver hvor meget overfladen bøjer en de forskellige retninger.
Da et ekstrema angiver en blob, ønskes der et punkt hvor de principielle krumninger er ens. For et punkt placeret på et maksima vil den principielle krumning være konkav (concave) og derfor negativ. For et punkt placeret på et minima vil den principielle krumning være konveks (convex) og derfor positiv. Hvor den for et saddle-point vil være positiv og negativ.
<scnd order deriv test> \\
<second order derivivative test multivar> \\
<Hessian>\\
<Sporret>\\
<Determinant>\\
Ved oprettelse af skalarum udglattes regioner i billedet, til grovere strukturer. Blev man bedt om at udpege strukturer på tværs af skalaer (bestående af grove strukturer) vil det være naturligt at udvælge lyse/mørke pletter (blobs), da de vil korrelere i placering og i struktur. 
<Konsistente over skala>

En blob er en region i et billede, hvor intensitet er konstant, og forskellig fra intensiteten udenfor regionen. Lindenberg \cite{blob} definere blobs som værende lyse regioner på sort baggrund eller omvendt - altså strukturer, der står i kontrast til deres baggrund. 

%MEGA FORKERT
%MEGA FORKERT
%MEGA FORKERT
%MEGA FORKERT
I figur \ref{fig:lindblob}(a), ses en blob defineret af dens lokale ekstrema, hvor styrken af blobben beksrives ved kontrasten, ift. området omkring ekstremaet. Lindenberg definere en blob som værende afgrænset af dens saddelpunkt; et saddelpunkt angiver punktet, hvor intensiteten stopper med at falde og starter med at stige for lyse blobs, og modsat for mørke.
%MEGA FORKERT
%MEGA FORKERT
%MEGA FORKERT
%MEGA FORKERT
%MEGA FORKERT
%MEGA FORKERT
%MEGA FORKERT
\\
\\
En metode til at detektere ekstremaer, er Laplace af Gauss(LoG). Laplace defineres således:
\begin{equation}
\Delta f = \nabla^2 f =  \sum_{i = 1}^n \frac{\partial^2 f}{\partial x^2_i}
\end{equation}
Anvendes Laplace operatoren på Gaussfunktionen (ligning \eqref{2dgaussian}), kan resultatet diskretiseres og anvendes som en kerne.
\begin{equation}
LoG= \sigma^2\Delta G
\label{lap}
\end{equation}
Bemærk, at $\sigma^2$ er blevet multipliceret på fra venstre. Dette er gjort for at normalisere LoG, således, at responset er invariant overfor størrelsen af $\sigma$. LoG kan diskritiseres og foldes med billedet for at finde lokale ekstremaer. \\

\begin{figure}[H]
    \centering
    \includegraphics[width=0.90\textwidth]{fig/normLoG.jpg}
    \vspace{-0.5em}   
    \begin{center}
    \caption{\textcolor{gray}{\footnotesize \textit{
    (a) En 3-D visualisering af en to-dimensional Laplacian of Guassian (b) ét en-dimensionalt signal (c) Laplacian of Gaussian operatoren anvendt på (b)}}}
    \label{fig:normLoG}
     \end{center}
  \end{figure}
       \vspace{-2.5em}
\noindent
Når blobs skal lokaliseres, skal $\sigma$ værdien være tilpasses blobbens størrelse, som ses på figur \eqref{fig:normLoG}. Figuren viser fire forskellige værdier for $\sigma$. I første illustration er $\sigma$ værdien for lav - her dannes flere ekstremaer, men ingen af dem karakteriserer en blob, da den absolutte værdi for funktionen evalueret på det afledte af LoG signalet, er for lav. Dog ser kurven i illustration tre ud til, at have tilstrækkelig høj absolut værdi, til at kunne karakteriseres som et blob. \\
Problemet omkring valg af skala, kan afhjælpes ved skalarumsrepræsentation, som gennemgås i.

\begin{figure}[H]
    \centering
    \includegraphics[width=0.25\textwidth]{fig/29.png}
    \vspace{-0.5em}   
    \begin{center}
    \caption{\textcolor{gray}{\footnotesize \textit{
    }}}
    \label{fig:scale}
     \end{center}
  \end{figure}
       \vspace{-2.5em}
\noindent
I figur \ref{fig:scale} angiver cirklerne forskellige undersøgte skalaer, Så hvordan udvælges cirklen, der dækker interesse området uafhængigt af områdets størrelse?  For Blobs er det interessant når der i et skaleret område opstår et veldefineret ekstrema. En måde at søge efter ekstremaer over forskellige skalaer er ved at oprette et skalarum for det undersøgte billede, hvor hvert billede skaleres og der for hver skala findes interessepunkter. Skala-rummet i et 2-dimensionalt billede repræsenteres af flere billeder i forskellige skalaer af det originale billede. Billeder, der repræsentere forskellige skalaer, opnås ved at folde billedet iterativt med et Gaussisk filter med stigende $\sigma$ værdi. 
\begin{figure}[H]
    \centering
    \includegraphics[width=0.65\textwidth]{fig/24.png}
    \vspace{-0.5em}   
    \begin{center}
    \caption{\textcolor{gray}{\footnotesize \textit{
Til venstre ses en visualisering af et skala-rum formet som en pyramide. Hvert niveau angiver en skala repræsentation af det originale vindue, hvor toppen af pyramiden indeholder billeder af største skala og derfor med mindst information, og bunden af skalaen med det originale billede. Til højre ses et billede foldet med et Gaussisk filter af stigende sigma værdier. Jo højere sigma værdi, jo flere fine detaljer bliver fjerne og billedet slørret.
    }}}
    \label{fig:mona}
     \end{center}
  \end{figure}
       \vspace{-2.5em}
\noindent
Et Gaussisk filter bruges da gradvis højere værdier af $\sigma$ fjerner fine strukturer, som vist i figur \ref{fig:mona}, og nye strukturer forekommer ikke ved transformationen fra finere til grovere skalaer \cite{lindenscale}. Idéen er derved at fjerne disse strukturer og lede efter  andre ekstremaer, gradvist på større skalaer, der også kan detekteres.
Et billede i skalarummet for billedet $f(x,y)$ kan derfor defineres som i \eqref{scalespace}
\begin{equation}
L(x,y,\sigma) = G(x,y,\sigma)\ast f(x,y)
\label{scalespace}
\end{equation}
hvor $G$ er det 2-dimensionelle Gaussiske filter,$L(x,y,\sigma)$ repræsentere et et billede i skala-rummet, og skala-parametren $\sigma$, bestemmer skalaen, eller placeringen i skala-rummet. $L(x,y,0) = f(x,y)$, da det er den "nederste" skala og den nederste del af pyramiden. Højere niveauer af pyramiden kan opnås ved at folde billedet med et Gaussisk filter af større sigma værdi.