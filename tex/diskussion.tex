\chapter{Diskussion}
I denne sektion vil resultater, fremført i sektion \ref{sec:resultater}, fortolkes og kommenteres, for at besvare de opstillede problemstillinger. \\
Først og fremmest var det interessant for dette projekt, at udlede hvorvidt det var muligt at opnå en korrekt korrespondanceanalyse, dvs. at udvælge korrekte korresponderende punkter i markbillederne. Derudover omhandler opgavens hovedproblemstilling, hvilke af de udvalgte metoder, der bedst muliggør en korrekt korrespondance analyse af markbillederne. For at kunne konkludere ovenstående, iagttages to aspekter fra resultaterne: (1) hvor mange korrekte korrespondancer, hver metode har fundet. (2) hvor mange korrekte korrespondancer, hver metode har fundet, ift. mængden af de detekterede interessepunkter. Disse to aspekter er opskrevet i figur \ref{table:tab} og opsummere opgavens resultater.
\begin{center}
    \begin{tabular}{ | l | l | l | l |}
    \hline
    Detektor & Deskriptor & $\text{Match(A,A')}$ & $\text{mean rm}$ \\ \hline
    $DoG$ & SURF & TBD & 0.2342 \\ \hline       
    Harris & SIFT & TBD & 1.505 \\ \hline    
    Moravec & SIFT & TBD & TBD \\ \hline    
    $DoH$ & SURF & TBD & 0.0292 \\ \hline    
    \end{tabular}
    \label{table:tab}
\end{center}
Figur \ref{table:tab} viser at det var muligt for alle metoder at udvælge korrekte korrespondancer i billederne og derved opnå en korrekt korrespondanceanalyse af markbillederne. For at besvare hovedproblemstillingen, evalueres metoderne udefra deres repeatability measure $rm$. Det kan ud fra dette konkluderes at metode kombinationen $DoG$ \& SURF, bedst muliggør en korrespondanceanalyse af markbillederne. \\ \\
For metoderne viste det sig at størstedelen af de brugbare punkter, i et stort omfang, blev udvalgt på én specifik oktav. Til det er metoderne kun anvendt på den oktav, der gav de bedste resultater. Denne beslutning er yderligere beggrundet ved at dronen tager billederne fra én bestemt højde, der forekommer altså ikke højdeændringer i mellem de afprøvede billeder. Derfor kan det konkluderes at selvom størstedelen af de brugbare punkter blev udvalgt på samme okav, var der behov for en skalarumspræsentation af markbillederne, for at kunne udvælge hvilken oktav, der gav de bedste resultater.
\\ \\
Som set i figur <>, var den ikke-rotationsinvariante version af SUFR, ikke i stand til at etablere korrekte korrespondancer, når billederne var roteret ift. hinanden. Det ses også at den rotationsinvariante deskriptor SURF, var i stand til dette. Ud fra de to billeder kan det konkluderes at, det er nødvendigt at deskriptoren er rotationsinvariant for at opnå en korrekt korrespondance analyse for lige præcis de to billeder. De to udvalgte billeder repræsentere  dog ikke resten af billedsættet, da der ikke forekommer en så markant rotation i de andre billeder. Derfor kan det siges at være et isoleret tilfælde at det er nødvendigt for deskriptoren at være rotationsinvariant. Det har også vist sig at være muligt at etablere et stort antal korrekte korrespondancer ved brug af $U-SURF$.
%Hjørner