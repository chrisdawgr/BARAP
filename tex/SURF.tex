\section{SURF}
SURF\footnote{SpeededUp Robust Features},introduceret af Herbert Bay et. al \cite{SURF} i 2006 er, ligesom SIFT, en samlet metode, bestående af en detektor og en deskriptor, der finder og beskriver blobs af forskellige størrelser. I forhold til SIFT, har SURF den fordel, at den beregningsmæssigt kan optimeres til at være betydeligt hurtigere end SIFT.
\subsection{Determinant of Hessian}
Determinant of Hessian $DoH$ udgør feature detektoren i SURF. $DoH$ er baseret på Hessian matricen, som udregnes for hvert punkt $p$ i et billede:
\begin{equation}
\mathcal{H}(p, \sigma) = 
 \begin{bmatrix}
 	L_{xx}(p, \sigma) & L_{xy}(p, \sigma) \\
 	L_{xy}(p, \sigma) & L_{yy}(p, \sigma) 
 \end{bmatrix}
 \label{hessianmatrix}
\end{equation}
hvor $\sigma$ angiver skala og $L_{xx}$ defineres som: 
\begin{equation}
L_{xx}(x,y, \sigma) = (\frac{\partial^2 }{\partial x^2 } G(x,y,\sigma)) * I
\label{lxx}
\end{equation}
og analogt for $L_{yy}, L_{xy}$. Bay et al. anvender approkismerede boksfiltre: $D_{xx}$, $D_{yy}$ og $D_{xy}$, der udelukkende består af værdierne $\lbrace-2,-1,0, 1\rbrace$ som ved brug af integralbilleder nedsætter antallet af beregninger drastisk. I denne implementering er disse boksfiltre ikke anvendt, men differentierede Gauss filtre er anvendt i stedet. Figur \ref{fig:lxxlyylxy} er en illustration af de afledte Gaussiske filtre, der er anvendt. Denne designbeslutning skyldes, at beregningerne af integralbilleder komplicerede programmet. Da 
boksfiltrene er en approksimation til de differentierede Gaussfiltre, der er anvendt, er denne tilgang korrekt.
\begin{figure}[H]
    \centering
    \includegraphics[width=0.75\textwidth]{fig/31.png}
     \vspace{-0.5em}
    \begin{center}    
       \caption{{\footnotesize \textit{De anvendte Gaussiske filtre, partielt differentieret i $xy,x^2$ og $y^2$.}}}
    \label{fig:lxxlyylxy}
     \end{center}
     \vspace{-2.5em}
  \end{figure} \noindent
I SURF oprettes et skalarum, der ligesom i SIFT opdeles i skalaer og oktaver. I stedet for at reducere billedernes størrelser som i SIFT, forøges størrelserne af de anvendte filtre. For hver oktav anvendes der, i denne implementering, fire billeder, foldet med fire forskellige filterstørrelser. For hver oktav bruges nr. 2 filterstørrelse, fra forrige oktav som startoktav (startende på størelse 9), og størrelsen imellem filtre, fordobles fra forrige oktav, som vist i tabel \ref{fig:secderivfiltersize}. Her skal det bemærkes, at de andenafledte filtre, skal have dobbelt størrelse, af de førsteafledte.
\begin{figure}[H]
    \centering
    \begin{center}    
    \begin{tabular}{ | l | l | l | l | l |}
    \hline
    oktav & filter str 1 & filter str 2 & filter str 3 & filter str 4 \\ \hline
    1 & 9 & 15 & 21 & 27 \\ \hline
  	2 & 15 & 27 & 39 & 51 \\ \hline
  	3 & 27 & 51 & 75 & 99 \\ \hline
  	4 & 51 & 99 & 147 & 195 \\ \hline
    \end{tabular}       
    \caption{{\footnotesize \textit{Fire forskellige oktaver, og filterstørrelse, for et andenafledt filter}}}
    \label{fig:secderivfiltersize}
     \end{center}
     \vspace{-2.5em}
  \end{figure} \noindent
For hvert punkt i billederne opstilles Hessian matricen, hvorefter determinanten udregnes for alle punkter i billederne. Bay et.al. udregner determinanten ved:
\begin{equation}
\textbf{det}\mathcal{H}_{approksimeret} = D_{xx}D_{yy}-wD_{xy}^2
\label{deerminantofhessian}
\end{equation}
hvor $w$ er en vægt, tilføjet for at balancere de approksimerede, differentierede Gaussiske filtre og boksfiltrene. Da denne implementering ikke anvender boksfiltre, udregnes determinanten som:
\begin{equation}
\textbf{det}\mathcal{H} = L_{xx}L_{yy}-L_{xy}^2
\label{deerminantofhessian}
\end{equation}
Når hessian matricens egenværdier har samme fortegn, indikere dette et ekstrema:
\begin{equation}
\begin{split}
\text{indikator} = 
\begin{cases}
\text{Ekstrema}& \text{hvis } \bold{det}(\mathcal{H}) > 0,  \\
\text{Saddel-Punkt} & \text{hvis } \bold{det}(\mathcal{H}) < 0.
\end{cases}
\end{split}
\label{detman}
\end{equation}
Et punkt udvælges derfor, hvis det er det største af et $3\times3\times3$ område af determinant billederne, som vist i figur \ref{fig:1difference}. Dette udføres for alle oktaver. Non-maximal suppression anvendes ligesom i SIFT.
\subsubsection*{Algoritme: SURF Detektor}
\begin{tabbing}
Input\quad \= : \= Billede $I$\\
Output \text{ } \> : \> Interessepunkter $p \in (x,y)$
\end{tabbing}
\begin{enumerate}
\item {Hessian matricen opstilles for alle punkter i skalabillederne som i ligning \eqref{hessianmatrix}}
\item Determinantbilleder opstilles, ved at udregne determinanten af Hessian matricen for alle punkter i alle billeder som i ligning \eqref{deerminantofhessian}.
\item Lokale maxima udvælges, af hvert $3\times3\times3$ område af determinantbillederne på samme oktav.
\item Non-maximal suppression, bruges til at fjerne dårligt lokaliserede punkter ligesom i ligning \eqref{maxsurp}.
\end{enumerate}
\subsection{Orientering}
\begin{figure}[H]
    \centering
    \includegraphics[width=0.75\textwidth]{fig/haarwavelet.png}
     \vspace{-1em}
    \begin{center}    
       \caption{\textcolor{gray}{\footnotesize \textit{ }}}
    \label{fig:haarwavelet}
     \end{center}
     \vspace{-2.5em}
  \end{figure} \noindent
SURF gør brug af Haar-wavelet, både i detektoren og deskriptoren. Haar-wavelet er et boks-filter, med størrelse nxn, hvor halvdelen af alle indgangene er +1 og den anden halvdel er -1.
\subsection{Deskriptor}
SURF deskriptoren producerer features, der består af 64 indgange. Den kan beskrives stringent, ligesom (26). Som beskrevet i SURF papiret af Tuytelaars et. al[??], er tilfældet ofte, at der ikke er brug for rotationsinvarians. En variant af SURF, der ikke er rotationsinvariant, kaldes Upright-SURF (U-SURF). Tuytelaars et. al garanterer dog rotationsinvarians i U-SURF, i op til $\pm 15^{\circ}$. SURF er implementeret, fremfor U-SURF, da enkelte af billederne har stor grad (ca. $15^{\circ}$) af rotation. Rotationen forekommer i billeder, der er taget lige efter, at dronen har vendt og kan ses på figur 15.
\\
\\
For at gøre SURF invariant overfor rotation, skal Haar-wavelet responset findes i x og y retningen omkring punktet ($dx, dy$), respektivt. $dx$ og $dy$ beregnes i en cirkel omkring interessepunktet, med radius $6\sigma$, og en afstand mellem punkterne, på $\sigma$. $dx$ og $dy$ foldes med en cirkulær Gausskerne, med samme størrelse som Haar-wavelet responsene, og $\sigma_{Gauss} = 2.5\sigma$.
\begin{figure}[H]
    \centering
    \includegraphics[width=0.2\textwidth]{fig/surforientation.jpg}
     \vspace{-1em}
    \begin{center}    
       \caption{\textcolor{gray}{\footnotesize \textit{ }}}
    \label{fig:surforientation}
     \end{center}
     \vspace{-2.5em}
  \end{figure} \noindent
Et sliding vindue på $60^{\circ}$, summerer alle vektorene, der ligger indenfor dets rækkevidde. Den længste vektor, udgør retningen, som beregnes med $atan2$ (invers tangens, med 2 inputs). Dette er illustreret på figur \ref{fig:surforientation}. Der er her taget en beslutning om, at slidingvinduet udregner summerne af vektorer, indenfor seks positioner ($0^{\circ}-60^{\circ}$, $60^{\circ}-120^{\circ}$.. $300^{\circ}-360^{\circ}$). 
\\
\\
SURF deskriptoren samler data omkring interssepunktet, med et dataindsamlingvindue der har størrelse $20 \sigma$. Vinduet skal orienters, ifht. den $\theta$ værdi, der er udregnet i detektoren. Billedet vendes, ved, at vende hele billedet, ved brug af ligning \eqref{rotaionmatrix}, og danne et integralbillede udfra det nye billede (Dette skridt er omkostningsfuldt, og der vil senere blive diskuteret optimeringer).
\\
\\
Dataindsamlingsvinduet er nu roteret og centreret omkring interessepunktet. Herefter skal vinduet deles op i 4x4 regioner, hver bestående af 5x5 punkter, der har ens afstand mellem sig - her er ikke blevet lavet subpixel-optimeringen, så hvis $\frac{20\sigma}{4}$ ikke er deleligt med 5, bliver der rundet ned.
\\
\\
Herefter skal Haar-wavelet responset udreges, i x og y retningen ($dx, dy$, respektivt). Haar-wavelet skal udregnes for alle punkter i 5x5 felterne. Størrelsen på Haar-wavelet filteret, er $2\sigma$. Disse værdier skal smoothes med et Gaussfilter, hvor $\sigma_{Gauss} = 3.3\sigma_{point}$.
\\
\\
For her af de 16 4x4 regioner, udregnes: 
\begin{equation}
v_i = \sum d_x, \sum d_y, \sum |d_y|, \sum |d_y|
\label{surffeature}
\end{equation}
hvor $v_i$ er beregningerne, for den $i$'ende region. Der laves en vektor, hvor alle værdierne indgange i $v$ sættes efter hinanden. Dette giver en $16 \cdot 4 = 64$ indgange stor vektor. Slutteligt, laves vektoren om til en enhedsvektor.

\subsubsection*{Algoritme}
\begin{enumerate}
\item Orienteringen findes: \begin{enumerate}

	\item Orienteringen på punktet findes, ved at beregne $dx$, $dy$ i et cirkulært område, med radius $6\sigma$, og afstand $\sigma$, mellem punkterne. 
	\item $dx$, $dy$ foldes med en Gauss kerne, med $\sigma_{Gauss} = 2.5\sigma $
	\item Den længste vektor, af summerede vektorer, der ligger indenfor $60^{\circ}$, bruges til at tildele orientering $\theta$ til interessepunktet.
\end{enumerate}
\item Et dataindsamlingsvindue roteret ifht. $\theta$, beregnet som \eqref{rotaionmatrix} med størrelse $20\sigma$ indsamler punkter, der er har afstand $\sigma$ mellem hinanden.
\item Haar-wavelet responset, $dx$, $dy$ udregnes for de indsamlede punkter, med en filterstørrelse på $2\sigma$, og $dx, dy$ foldes med Gauss, hvor $\sigma_{Gauss} = 3.3\sigma$
\item For hver af det 4x4 regioner, udregnes \eqref{surffeature}. Dette sættes sammen til en vektor, der vil have 64 indgange
\end{enumerate}